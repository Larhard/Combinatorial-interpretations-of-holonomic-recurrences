\documentclass[final]{beamer}
\RequirePackage{beamertransparentboxes}
\RequirePackage{pgfpages}

\usefonttheme[onlymath]{serif}

% \documentclass[handout,final]{article}
% \RequirePackage{beamerarticle}

\setbeamerfont{note page}{size=\scriptsize}
\addtobeamertemplate{note page}{\setbeamerfont{itemize/enumerate subbody}{size=\scriptsize}\setlength{\parindent}{0pt}\setlength{\parskip}{.7em}}{\par}

\mode<handout>{%
    \pgfpagesuselayout{2 on 1}[a4paper] 
    \setbeameroption{show notes on second screen=bottom}
}

\usetheme{Warsaw}

\RequirePackage[utf8]{inputenc}
\DeclareUnicodeCharacter{00A0}{ }

\RequirePackage{color}
\definecolor{ao(english)}{rgb}{0.0, 0.5, 0.0}
\definecolor{blue(pigment)}{rgb}{0.2, 0.2, 0.6}
\definecolor{egyptianblue}{rgb}{0.06, 0.2, 0.65}

\RequirePackage{listings}
\lstset{
    basicstyle=\scriptsize
}

\RequirePackage{graphicx}
\RequirePackage{svg}
\RequirePackage{float}

\svgpath{../article/images/}

\RequirePackage{amsthm}
\RequirePackage{amsmath}
\RequirePackage{mathtools}
\RequirePackage{amsfonts}
\RequirePackage{soul}
\RequirePackage{xspace}
\RequirePackage{bussproofs}
\RequirePackage[customcolors]{hf-tikz}

\RequirePackage{tikz}

\RequirePackage{multicol}
\RequirePackage{lipsum}
\RequirePackage{textpos}

\setlength{\parindent}{0pt}
\setlength{\parskip}{.7em}

\RequirePackage{polski}

\uselanguage{polski}
\languagepath{polski}
\deftranslation[to=polski]{Definition}{Definicja}

\logo{\includegraphics[height=.3\textheight]{logo.png}}

\title{Kombinatoryczne interpretacje rekurencji holonomicznych}
\author[Bartłomiej Puget (TCS UJ)]{Bartłomiej Puget\\Promotor: dr Katarzyna Grygiel}
\institute{Theoretical Computer Science\\Jagiellonian University}

\date{Kraków, 19 maja 2021}

\makeatletter
\def\th@bluetheorem{%
    \normalfont % body font
    \setbeamercolor{block title example}{bg=blue(pigment),fg=white}
    \setbeamercolor{block body example}{bg=blue(pigment)!20,fg=black}
    \setbeamercolor{item}{fg=blue(pigment)}
    \def\inserttheoremblockenv{exampleblock}
}
\def\th@greentheorem{%
    \normalfont % body font
    \setbeamercolor{block title example}{bg=ao(english),fg=white}
    \setbeamercolor{block body example}{bg=ao(english)!20,fg=black}
    \setbeamercolor{item}{fg=ao(english)}
    \def\inserttheoremblockenv{exampleblock}
}
\makeatother

\theoremstyle{bluetheorem}
\newtheorem{mytheorem}{Definicja}[section]

\theoremstyle{bluetheorem}
\newtheorem{mydefinition}[mytheorem]{Definicja}
\newtheorem{myproductions}[mytheorem]{Produkcje}

\theoremstyle{greentheorem}
\newtheorem{myexample}[mytheorem]{Przykład}

\newcounter{highlightid}
\newcommand{\mhl}[1]{{\tikzmarkin[set fill color=red!30,set border color=red]{highlight-\arabic{highlightid}}(0.02,-0.15)(-0.02,0.35){#1}\tikzmarkend{highlight-\arabic{highlightid}}\stepcounter{highlightid}}}
\newcommand{\mhll}[1]{{\tikzmarkin[set fill color=red!30,set border color=red]{highlight-\arabic{highlightid}}(0.15,-0.15)(-0.15,0.35){#1}\tikzmarkend{highlight-\arabic{highlightid}}\stepcounter{highlightid}}}

\newcommand{\mhlr}[1]{{\tikzmarkin[set fill color=red!30,set border color=red]{highlight-\arabic{highlightid}}(0.02,-0.15)(-0.02,0.35){#1}\tikzmarkend{highlight-\arabic{highlightid}}\stepcounter{highlightid}}}
\newcommand{\mhlg}[1]{{\tikzmarkin[set fill color=green!30,set border color=green]{highlight-\arabic{highlightid}}(0.02,-0.15)(-0.02,0.35){#1}\tikzmarkend{highlight-\arabic{highlightid}}\stepcounter{highlightid}}}
\newcommand{\mhlb}[1]{{\tikzmarkin[set fill color=blue!30,set border color=blue]{highlight-\arabic{highlightid}}(0.02,-0.15)(-0.02,0.35){#1}\tikzmarkend{highlight-\arabic{highlightid}}\stepcounter{highlightid}}}

\newcommand{\gf}[1]{\ensuremath{\mathcal{#1}}}
\newcommand{\enc}[1]{\ensuremath{\ols{#1}}}
\newcommand{\pointed}[1]{\ensuremath{{#1}^*}}
\newcommand{\weighted}[1]{\ensuremath{_{(#1)}}}
\newcommand{\size}[1]{\ensuremath{\left|#1\right|}}
\newcommand{\LambdadB}{\ensuremath{\Lambda_{\text{dB}}}}

\DeclareMathOperator{\textiff}{\text{iff}}
\DeclareMathOperator{\N}{\mathbb{N}}
\DeclareMathOperator{\poly}{\mathbb{P}}
\DeclareMathOperator{\troot}{\text{root}}
\DeclareMathOperator{\tleft}{\text{left}}
\DeclareMathOperator{\tright}{\text{right}}
\DeclareMathOperator{\tLeft}{\text{Left}}
\DeclareMathOperator{\tRight}{\text{Right}}
\DeclareMathOperator{\tSub}{\text{Sub}}
\DeclareMathOperator{\tPre}{\text{Pre}}
\DeclareMathOperator{\tSuf}{\text{Suf}}
\DeclareMathOperator{\tClass}{\text{Class}}
\DeclareMathOperator{\tSection}{\text{Section}}
\DeclareMathOperator{\tPart}{\text{Part}}
\DeclareMathOperator{\n}{\bullet}
\DeclareMathOperator{\no}{\o}

\newcommand{\includeinlinesvg}[2]{\begin{minipage}{#1\textwidth}\includesvg[width=\textwidth]{#2}\end{minipage}}
\newcommand{\includeinlinescaledsvg}[3]{\begin{minipage}{#1\textwidth}\begin{center}\includesvg[scale=#2]{#3}\end{center}\end{minipage}}

\begin{document}
\maketitle

\section{Wstęp}

\begin{frame}{Rekurencje holonomiczne}
    \begin{mydefinition}
        Ciąg \((a_n)_{n=0}^\infty\) jest P-rekurencyjny (holonomiczny) wtedy i tylko wtedy, gdy istnieje dla niego rekurencja holonomiczna, tj.:
        \[\exists_{r \in \mathbb{N}} \exists {p_0, \ldots, p_r : \text{ poly }}\forall_{n \in \mathbb{N}}:\]
        \[p_0(n) a_n + p_1(n) a_{n + 1} + \ldots + p_r(n) a_{n + r} = 0\]
    \end{mydefinition}

    \begin{myexample}
        \(a_n = \frac{5 n - 3}{3 n + 5}\) jest holonomiczny, gdyż zachodzi:
        \[(3n + 5) (5n + 2) a_n - (5n - 3)(3n + 8) a_{n + 1} = 0\]
    \end{myexample}
\end{frame}

\begin{frame}{Rekurencje holonomiczne (równoważnie)}
    \begin{mydefinition}
        Ciąg \((a_n)_{n=0}^\infty\) jest P-rekurencyjny (holonomiczny) wtedy i tylko wtedy, gdy istnieje dla niego funkcja tworząca:
        \[p_0(z) C(z) + p_1(z) C'(z) + \ldots + p_r(z) C^{(r - 1)}(z) = b(z)\]
        dla pewnych:\\
        \(r \in \mathbb{N}\)\\
        \(b, p_0, \ldots, p_r : \text{ poly }\)
    \end{mydefinition}
\end{frame}

\begin{frame}{Rekurencje holonomiczne}
    \begin{center}
        \begin{minipage}[t]{.2\textwidth}
            \begin{center}
                \(B(z)\)\\\bigskip
                \includesvg[scale=0.4]{binary__def_1}%
            \end{center}
        \end{minipage}%
        \begin{minipage}[t]{.05\textwidth}
            \begin{center}
                \(=\)
            \end{center}
        \end{minipage}%
        \begin{minipage}[t]{.2\textwidth}
            \begin{center}
                \(1\)\\\bigskip
                \includesvg[scale=0.4]{binary__def_2}%
            \end{center}
        \end{minipage}%
        \begin{minipage}[t]{.05\textwidth}
            \begin{center}
                \(+\)\\
            \end{center}
        \end{minipage}%
        \begin{minipage}[t]{.2\textwidth}
            \begin{center}
                \(z B^2(z)\)\\\bigskip
                \includesvg[scale=0.4]{binary__def_3}%
            \end{center}
        \end{minipage}%
    \end{center}
    \[\left\{\begin{array}{rcl}
                b_0 &=& 1\\
                (n + 2)b_{n + 1} &=& 2 (2n + 1)b_n
    \end{array}\right.\]
\end{frame}

\begin{frame}{Interpretacje kombinatoryczne}
    \begin{myexample}
        \[(n + 2)b_{n + 1} = 2 (2n + 1)b_n\]

        \begin{center}
            \small
            \begin{minipage}{.20\textwidth}\includesvg[scale=0.30]{binary__remy_base}\end{minipage}%
            \(\Rightarrow\)
            \begin{minipage}{.3\textwidth}\includesvg[scale=0.40]{binary__remy_left}\end{minipage}%
            \begin{minipage}{.3\textwidth}\includesvg[scale=0.40]{binary__remy_right}\end{minipage}%
        \end{center}
    \end{myexample}
\end{frame}

\section{Problem}

\begin{frame}{Drzewa termów lambda w unarnej notacji de Bruijna}
    \begin{columns}
        \column{.6\textwidth}
        \begin{mydefinition}
            \begin{itemize}
                \item \(\forall n \in \N \implies n \in \LambdadB\)
                \item \(M \in \LambdadB \implies (\lambda M) \in \LambdadB\)
                \item \(M, N \in \LambdadB \implies (M N) \in \LambdadB\)
            \end{itemize}
        \end{mydefinition}

        \column{.4\textwidth}
        \begin{myexample}
            \begin{center}
                \includeinlinescaledsvg{1}{0.6}{lambda__tree_structure__deBruijn_002_no_green_custom}
            \end{center}
        \end{myexample}
    \end{columns}
\end{frame}

\begin{frame}{Drzewa termów lambda w unarnej notacji de Bruijna}
    \begin{myexample}
        \begin{columns}
            \column{.475\textwidth}
            \begin{center}
                \includeinlinescaledsvg{1}{0.6}{lambda__tree_structure__deBruijn_002_no_green_custom}
            \end{center}
            \column{.05\textwidth}
            \begin{center}
                    \raisebox{.25\textwidth}{$\equiv$}
            \end{center}
            \column{.475\textwidth}
            \begin{center}
                \includeinlinescaledsvg{1}{0.6}{lambda__tree_structure__deBruijn_002_custom}
            \end{center}
        \end{columns}
    \end{myexample}
\end{frame}

\begin{frame}{Rekurencja holonomiczna}
    \begin{columns}
        \column{.6\textwidth}
        \begin{mydefinition}
            \[\begin{array}{rl}
                    0 =& (n - 4) t_{n - 4}\\
                    +& t_{n - 3}\\
                    +& (2 n - 1) t_{n - 2}\\
                    +& (-4 n + 1) t_{n - 1}\\
                    +& (n + 1) t_{n}.
            \end{array}\]
        \end{mydefinition}

        \column{.4\textwidth}
        \begin{myexample}
            \begin{center}
                \includeinlinescaledsvg{1}{0.6}{lambda__tree_structure__deBruijn_002_no_pointers}
            \end{center}
        \end{myexample}
    \end{columns}
\end{frame}

\section{Podejście do problemu}

\begin{frame}{Rekurencja holonomiczna}
    \begin{mydefinition}
        \begin{center}
            \begin{minipage}[t]{.2\textwidth}
                \begin{center}
                    \(\gf{T}\)\\
                    \includesvg[scale=0.3]{lambda__def__1}%
                \end{center}
            \end{minipage}%
            \begin{minipage}[t]{.05\textwidth}
                \begin{center}
                    \(=\)\\
                \end{center}
            \end{minipage}%
            \begin{minipage}[t]{.2\textwidth}
                \begin{center}
                    \(z a~\gf{T}^2\)\\
                    \includesvg[scale=0.3]{lambda__def__2}%
                \end{center}
            \end{minipage}%
            \begin{minipage}[t]{.05\textwidth}
                \begin{center}
                    \(+\)\\
                \end{center}
            \end{minipage}%
            \begin{minipage}[t]{.2\textwidth}
                \begin{center}
                    \(z l \gf{T}\)\\
                    \includesvg[scale=0.3]{lambda__def__3}%
                \end{center}
            \end{minipage}%
            \begin{minipage}[t]{.05\textwidth}
                \begin{center}
                    \(+\)\\
                \end{center}
            \end{minipage}%
            \begin{minipage}[t]{.2\textwidth}
                \begin{center}
                    \(\gf{V}\)\\
                    \includesvg[scale=0.3]{lambda__def__4}%
                \end{center}
            \end{minipage}%
        \end{center}

        \begin{center}
            \begin{minipage}[t]{.2\textwidth}
                \begin{center}
                    \(\gf{V}\)\\
                    \includesvg[scale=0.3]{lambda__def__5}%
                \end{center}
            \end{minipage}%
            \begin{minipage}[t]{.05\textwidth}
                \begin{center}
                    \(=\)\\
                \end{center}
            \end{minipage}%
            \begin{minipage}[t]{.2\textwidth}
                \begin{center}
                    \(z s \gf{V}\)\\
                    \includesvg[scale=0.3]{lambda__def__7}%
                \end{center}
            \end{minipage}%
            \begin{minipage}[t]{.05\textwidth}
                \begin{center}
                    \(+\)\\
                \end{center}
            \end{minipage}%
            \begin{minipage}[t]{.2\textwidth}
                \begin{center}
                    \(z o\)\\
                    \includesvg[scale=0.3]{lambda__def__6}%
                \end{center}
            \end{minipage}%
        \end{center}
    \end{mydefinition}
\end{frame}

\begin{frame}{Rekurencja holonomiczna}
    \begin{mydefinition}
        \[\begin{array}{rl}
                0 =& (n - 4) t_{n - 4}\\
                +& t_{n - 3}\\
                +& (2 n - 1) t_{n - 2}\\
                +& (-4 n + 1) t_{n - 1}\\
                +& (n + 1) t_{n}.
        \end{array}\]
    \end{mydefinition}

    \begin{mydefinition}
        \[\begin{array}{rl}
            0 =& (n - 4) l^2 s^2 t_{n - 4}\\
            +& ((4 n - 10) a o s - (2 n - 6) l^2 s - (2 n - 5) l s^2) t_{n - 3}\\
            +& ((-4 n + 8) a o + (n - 2) l^2 + (4 n - 6) l s + (n - 1) s^2) t_{n - 2}\\
            +& ((-2 n + 1) l - 2 n s) t_{n - 1}\\
            +& (n + 1) t_{n}.
        \end{array}\]
    \end{mydefinition}
\end{frame}

\begin{frame}{Sekcje}
    \begin{block}{}
        \[\begin{array}{lcr}
            \tSection_{\emptyset} &:& (k + 1) t_k\\
            \tSection_{\{l\}} &:& - (2 k + 1) t_k\\
            \tSection_{\{s\}} &:& - 2 (k + 1) t_k\\
            \tSection_{\{ao\}} &:& - (4 k) t_k\\
            \tSection_{\{ll\}} &:& (k) t_k\\
            \tSection_{\{ls\}} &:& (4 k + 2) t_k\\
            \tSection_{\{ss\}} &:& (k + 1) t_k\\
            \tSection_{\{aos\}} &:& 2 (2 k + 1) t_k\\
            \tSection_{\{lls\}} &:& - (2 k) t_k\\
            \tSection_{\{lss\}} &:& - (2 k + 1) t_k\\
            \tSection_{\{llss\}} &:& (k) t_k\\
        \end{array}\]
    \end{block}
\end{frame}

\begin{frame}{Transformacje kontekstów}
    \begin{block}{}
        \begin{center}
            \includeinlinescaledsvg{.4}{.5}{lambda__contexts__def_001}%
            \(\Rightarrow\)
            \includeinlinescaledsvg{.4}{.5}{lambda__contexts__def_002}%
        \end{center}
    \end{block}
\end{frame}

\begin{frame}{Konteksty startowe}
    \begin{block}{}
        \begin{center}
            \includeinlinescaledsvg{.33}{.35}{lambda__contexts__type_001}%
            \includeinlinescaledsvg{.33}{.35}{lambda__contexts__type_002}%
            \includeinlinescaledsvg{.33}{.35}{lambda__contexts__type_003}\\%
            \includeinlinescaledsvg{.33}{.35}{lambda__contexts__type_004}%
            \includeinlinescaledsvg{.33}{.35}{lambda__contexts__type_004b}\\%
            \includeinlinescaledsvg{.33}{.35}{lambda__contexts__type_005}%
            \includeinlinescaledsvg{.33}{.35}{lambda__contexts__type_005b}%
        \end{center}
    \end{block}
\end{frame}

\begin{frame}[fragile]{Iteracyjne drążenie tematu}
    \begin{block}{}
\begin{lstlisting}
=== Classes stats ===
[ ]  0,0:   0 of    16 | (k + 1) * Tk           []
[ ]  1,0:   0 of   -10 | - (2 * k + 1) * Tk     [l]
[ ]  1,1:   0 of   -12 | - 2 * (k + 1) * Tk     [s]
[ ]  2,0:   0 of    -4 | - (4 * k) * Tk         [ao]
[ ]  2,1:   0 of     1 | (k) * Tk               [ll]
[ ]  2,2:   0 of     6 | (4 * k + 2) * Tk       [ls]
[ ]  2,3:   0 of     2 | (k + 1) * Tk           [ss]
[ ]  3,0:   0 of     2 | 2 * (2 * k + 1) * Tk   [aos]
[x]  3,1:   0 of     0 | - (2 * k) * Tk         [lls]
[ ]  3,2:   0 of    -1 | - (2 * k + 1) * Tk     [lss]
[x]  4,0:   0 of     0 | k * Tk                 [llss]
\end{lstlisting}
    \end{block}
\end{frame}

\begin{frame}[fragile]{Iteracyjne drążenie tematu}
    \begin{block}{}
        \begin{center}
            \includeinlinescaledsvg{.4}{.4}{lambda__transformations__001a}%
            \(\xRightarrow{+1}\)%
            \includeinlinescaledsvg{.4}{.4}{lambda__transformations__001b}%
        \end{center}
    \end{block}
\end{frame}

\begin{frame}[fragile]{Iteracyjne drążenie tematu}
    \begin{block}{}
\begin{lstlisting}
# define 0,0: (k + 1) * Tk []
=== Classes stats ===
[x]  0,0:   16 of    16 | (k + 1) * Tk           []
[ ]  1,0:    0 of   -10 | - (2 * k + 1) * Tk     [l]
[ ]  1,1:    0 of   -12 | - 2 * (k + 1) * Tk     [s]
[ ]  2,0:    0 of    -4 | - (4 * k) * Tk         [ao]
[ ]  2,1:    0 of     1 | (k) * Tk               [ll]
[ ]  2,2:    0 of     6 | (4 * k + 2) * Tk       [ls]
[ ]  2,3:    0 of     2 | (k + 1) * Tk           [ss]
[ ]  3,0:    0 of     2 | 2 * (2 * k + 1) * Tk   [aos]
[x]  3,1:    0 of     0 | - (2 * k) * Tk         [lls]
[ ]  3,2:    0 of    -1 | - (2 * k + 1) * Tk     [lss]
[x]  4,0:    0 of     0 | k * Tk                 [llss]
=== Diff stats ===
0,0:     0    16
\end{lstlisting}
    \end{block}
\end{frame}

\begin{frame}[fragile]{Iteracyjne drążenie tematu}
    \begin{block}{}
        \begin{center}
            \includeinlinescaledsvg{.4}{.4}{lambda__transformations__002a}%
            \(\xRightarrow{-1}\)%
            \includeinlinescaledsvg{.4}{.4}{lambda__transformations__002b}%
        \end{center}
    \end{block}
\end{frame}

\begin{frame}[fragile]{Iteracyjne drążenie tematu}
    \begin{block}{}
\begin{lstlisting}
=== Classes stats ===
[x]  0,0:   39050 of   39050 | (k + 1) * Tk          []
[x]  1,0:  -23237 of  -23237 | - (2 * k + 1) * Tk    [l]
[ ]  1,1:  -17799 of  -24460 | - 2 * (k + 1) * Tk    [s]
[x]  2,0:  -13728 of  -13728 | - (4 * k) * Tk        [ao]
[x]  2,1:    3432 of    3432 | (k) * Tk              [ll]
[ ]  2,2:   10833 of   14586 | (4 * k + 2) * Tk      [ls]
[ ]  2,3:    1770 of    3861 | (k + 1) * Tk          [ss]
[ ]  3,0:    2280 of    4620 | 2 * (2 * k + 1) * Tk  [aos]
[ ]  3,1:   -1648 of   -2156 | - (2 * k) * Tk        [lls]
[ ]  3,2:   -1140 of   -2310 | - (2 * k + 1) * Tk    [lss]
[ ]  4,0:     187 of     342 | k * Tk                [llss]
=== Diff stats ===
\end{lstlisting}
    \end{block}
\end{frame}


\section*{}

\begin{frame}{Pytania}
\end{frame}

\end{document}
