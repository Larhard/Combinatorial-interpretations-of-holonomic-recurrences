\documentclass[final]{article}

\usepackage{todonotes}
\setlength{\marginparwidth}{2.5cm}

\RequirePackage[utf8]{inputenc}
\DeclareUnicodeCharacter{00A0}{ }
\DeclareUnicodeCharacter{25CF}{\(\n\)}

\RequirePackage{color}
\definecolor{ao(english)}{rgb}{0.0, 0.5, 0.0}
\definecolor{blue(pigment)}{rgb}{0.2, 0.2, 0.6}
\definecolor{egyptianblue}{rgb}{0.06, 0.2, 0.65}
\definecolor{lightgray}{gray}{0.60}

\RequirePackage{listings}
\lstset{
    basicstyle=\footnotesize\ttfamily,
    comment=[l]{\#},
    morecomment=[s]{/*}{*/},
    commentstyle=\color{lightgray}\ttfamily,
    escapeinside={<@}{@>}
}

\RequirePackage{hyperref}
\hypersetup{
    colorlinks=true,
    linkcolor=black,
    citecolor=black,
    filecolor=black,
    urlcolor=black
}

\RequirePackage{geometry}
\RequirePackage{tabularx}
\RequirePackage{longtable}
\RequirePackage{graphicx}
\RequirePackage{svg}
\RequirePackage{float}

\svgpath{images/}

\RequirePackage[maxbibnames=99]{biblatex}
\addbibresource{bibliography.bib}

\RequirePackage{amsthm}
\RequirePackage{amsmath}
\RequirePackage{mathtools}
\RequirePackage{amsfonts}
\RequirePackage{soul}
\RequirePackage{xspace}
\RequirePackage{bussproofs}
\RequirePackage[customcolors]{hf-tikz}

\RequirePackage{tikz}
\RequirePackage{setspace}

\RequirePackage{multicol}
\RequirePackage{lipsum}
\RequirePackage{textpos}

\setlength{\parindent}{0pt}
\setlength{\parskip}{.7em}

\title{Combinatorial interpretations of~holonomic recurrences}
\author{Bartłomiej Puget}

\date{\today}

\theoremstyle{definition}
\newtheorem{definition}{Definition}[subsection]

\theoremstyle{definition}
\newtheorem{interpretation}{Interpretation}[subsection]

\theoremstyle{remark}
\newtheorem{example}{Example}[subsection]

\newcounter{highlightid}
\newcommand{\mhl}[1]{{\tikzmarkin[set fill color=red!30,set border color=red]{highlight-\arabic{highlightid}}(0.02,-0.15)(-0.02,0.35){#1}\tikzmarkend{highlight-\arabic{highlightid}}\stepcounter{highlightid}}}
\newcommand{\mhll}[1]{{\tikzmarkin[set fill color=red!30,set border color=red]{highlight-\arabic{highlightid}}(0.15,-0.15)(-0.15,0.35){#1}\tikzmarkend{highlight-\arabic{highlightid}}\stepcounter{highlightid}}}

\newcommand{\mhlr}[1]{{\tikzmarkin[set fill color=red!30,set border color=red]{highlight-\arabic{highlightid}}(0.02,-0.15)(-0.02,0.35){#1}\tikzmarkend{highlight-\arabic{highlightid}}\stepcounter{highlightid}}}
\newcommand{\mhlg}[1]{{\tikzmarkin[set fill color=green!30,set border color=green]{highlight-\arabic{highlightid}}(0.02,-0.15)(-0.02,0.35){#1}\tikzmarkend{highlight-\arabic{highlightid}}\stepcounter{highlightid}}}
\newcommand{\mhlb}[1]{{\tikzmarkin[set fill color=blue!30,set border color=blue]{highlight-\arabic{highlightid}}(0.02,-0.15)(-0.02,0.35){#1}\tikzmarkend{highlight-\arabic{highlightid}}\stepcounter{highlightid}}}

\newcommand{\ols}[1]{\mskip.5\thinmuskip\overline{\mskip-.5\thinmuskip {#1} \mskip-.5\thinmuskip}\mskip.5\thinmuskip} % overline short
\newcommand{\olsi}[1]{\,\overline{\!{#1}}} % overline short italic

\newcommand{\gf}[1]{\ensuremath{\mathcal{#1}}}
\newcommand{\enc}[1]{\ensuremath{\ols{#1}}}
\newcommand{\pointed}[1]{\ensuremath{{#1}^*}}
\newcommand{\weighted}[1]{\ensuremath{_{(#1)}}}
\newcommand{\size}[1]{\ensuremath{\left|#1\right|}}
\newcommand{\LambdadB}{\ensuremath{\Lambda_{\text{dB}}}}

\DeclareMathOperator{\textiff}{\text{iff}}
\DeclareMathOperator{\N}{\mathbb{N}}
\DeclareMathOperator{\poly}{\mathbb{P}}
\DeclareMathOperator{\troot}{\text{root}}
\DeclareMathOperator{\tleft}{\text{left}}
\DeclareMathOperator{\tright}{\text{right}}
\DeclareMathOperator{\tLeft}{\text{Left}}
\DeclareMathOperator{\tRight}{\text{Right}}
\DeclareMathOperator{\tSub}{\text{Sub}}
\DeclareMathOperator{\tPre}{\text{Pre}}
\DeclareMathOperator{\tSuf}{\text{Suf}}
\DeclareMathOperator{\tClass}{\text{Class}}
\DeclareMathOperator{\tSection}{\text{Section}}
\DeclareMathOperator{\tPart}{\text{Part}}
\DeclareMathOperator{\n}{\bullet}
\DeclareMathOperator{\no}{\o}

\newcommand{\includeinlinesvg}[2]{\begin{minipage}{#1\textwidth}\includesvg[width=\textwidth]{#2}\end{minipage}}
\newcommand{\includeinlinescaledsvg}[3]{\begin{minipage}{#1\textwidth}\begin{center}\includesvg[scale=#2]{#3}\end{center}\end{minipage}}

\begin{document}

\begin{titlepage}
	\begin{center}
	\textsc{\LARGE Jagiellonian University}

	\Large Faculty of Mathematics and Computer Science

	\Large Theoretical Computer Science

	\vfill

	\vspace{1cm}
	\hrulefill
	\vspace{0.5cm}

    \makeatletter
    \huge \textsc{\@title}
    \makeatother

	\vspace{0.2cm}
	\hrulefill

	\vspace{1cm}
    \makeatletter
	\textsc{\Large \@author}
    \makeatother

	\vspace{1cm}
    \normalsize

	Master Thesis\\
	Advisor: \textsc{Katarzyna Grygiel}

	\vfill

    \makeatletter
    \@date
    \makeatother
	\end{center}
\end{titlepage}

\section*{Abstract}%
\label{sec:abstract}

We analyze the problem of finding combinatorial interpretations of holonomic recurrences and show some techniques that were used to achieve known results.

We also try to find a~combinatorial interpretation of the holonomic recurrence for trees of lambda terms in the de Bruijn notation.

\clearpage

\tableofcontents
\clearpage

\section{Overview}%

Generating functions is a way of describing infinite sequences using formal power series, where coefficients encode the consecutive elements. It allows us to incorporate many advanced methods of analysis of encoded objects.

Most recursive structure can be described using a~generating function equation, which can be then solved and transformed into a~holonomic recurrence, i.e.~recurrence that in order to define \(n\)-th element of the sequence requires just finite number of previous ones and coefficients that are polynomials in \(n\).

It allows us to quickly count objects of given structure with given size. At first glance, the recurrence seems to be closely related to the structure itself. It just seems like it should describe simple transformations resulting in constructing bigger objects from slightly smaller ones. In case of trees it could be for example adding nodes. However, there is no known method of interpreting it in the language of the combinatorial structures.

Up to today only a few interpretations of holonomic recurrences on trees have been found, including interpretation of holonomic recurrences for binary trees~\cite{binary}, unary-binary trees~\cite{motzkin}, and Schröder trees~\cite{schroder}. Unfortunately, the mentioned interpretations do not come with a method of finding them that could be generalized to more complex cases.

The initial inspiration for the thesis came from the publication \textit{A natural counting of lambda terms}~\cite{inspiration}, which shows bijections among seemingly different tree structures, which are lambda trees, black-white trees~\cite{blackwhite} and zigzag free trees~\cite{inspiration}\todo{Find better cite}. They share the same holonomic recurrence, but there is no known immediate connection between the holonomic recurrence and the structures themselves.

Lambda terms are very popular structures used for analyzing functions and programming languages. Their very simple recursive definition makes them very attractive. A better understanding of them could allow us improving programming language compilers and in consequence improve every aspect of our lives. The interpretation of holonomic recurrence for lambda trees itself could allow us to discover new ways of generating random lambda terms via extending smaller ones in similar way to Rémy's algorithm for sampling binary trees~\cite{remy,note}.

Throughout the thesis we try to accumulate known results and techniques that may help solving the problem of finding combinatorial interpretations. We also propose an~approach that helped us to find an~intermediate solution for the problem of finding combinatorial interpretation for holonomic recurrence of lambda trees in the unary de Bruijn notation. We use the de Bruijn indices in order to get rid of letter-based variable names. Unary notation allows us to connect size of the tree to the values of the de Bruijn indices.

In Section~\ref{sec:introduction} we establish notation and introduce concepts that will be used throughout the thesis.

Next, in Section~\ref{sec:known_results}, we present known results, i.e.~interpretations for binary trees, unary-binary trees and Schröder trees. They demonstrate few different approaches to the problem.

In Section~\ref{sec:approaches} we show approaches to the problem that helped to achieve the aforementioned results. All of them were tried in order to find the answer to the original problem of interpreting holonomic recurrence of trees of lambda terms in the unary de Bruijn notation. Unfortunately, none of them helped us enough to get the final result. Anyway, they are proven to be helpful when we are stuck, as most of them are able to divide problems into smaller parts and detect corner cases.

Section~\ref{sec:brutal_iterative_approach} presents our final approach to the original problem, i.e.~computer-aided iterative approach that allowed us to find the intermediate solution. We believe that it is possible to extend this idea in order to obtain the full solution, but we were not able to achieve this due to the limited time that was available to us.

Finally, in Section~\ref{sec:further_work}, we present our view on what can be done with the problem in the future. We believe that one day the problem will be fully solved and a~more generic approach to interpreting holonomic recurrences will emerge.

\clearpage

\section{Introduction}%
\label{sec:introduction}

\subsection{Notation}%
\label{sub:notation}

In order to make reading this thesis easier, let's fix some notation and make some assumptions. Any deviation from this will be clearly stated. If you have trouble guessing what some letter or word means, this should be your primary place to look for the answer.

e.g.~stands for Latin exempli gratia, which means \textit{for example}.

i.e.~stands for Latin id est, which means \textit{that is}.

\(\textiff\) means if and only if.

\(\poly\) is a~set of all polynomials.

\(n, m, i, j, k\) are implicitly assumed to be natural numbers.

\(f, g\) are implicitly assumed to be functions.

\(f^{(n)}\) is implicitly defined as \(n\)-th derivative of function \(f\).

\(V(T)\) is implicitly defined as a~set of nodes of the tree \(T\).

\(E(T)\) is implicitly defined as a~set of edges of the tree \(T\).

\(u, v\) are implicitly assumed to be nodes.

\(e\) is implicitly assumed to be an~edge.

\(\troot(T)\) is implicitly defined as the~root of the~tree \(T\).

\(\tleft(v)\) is implicitly defined as the~left child of the~node \(v\).

\(\tright(v)\) is implicitly defined as the~right child of the~node \(v\).

\(\tSub(v)\) is implicitly defined as the~whole subtree rooted with the~node \(v\).

\(\tLeft(v)\) is implicitly defined as the~left subtree of the~node \(v\).

\(\tRight(v)\) is implicitly defined as the~right subtree of the~node \(v\).

Sibling of the node \(v\) is defined as the node \(u\) having the same parent as \(v\).

Uncle of the node \(v\) is defined as the~sibling of the parent of \(v\).

We say that node \(v\) is an~oneling \(\textiff\) \(v\) has no siblings. Note that the root is oneling even if it has no parent.

We say that node \(v\) is a~twinling \(\textiff\) \(v\) has at least one sibling.

Leaf is a~node which has no children.

\(\n\) is implicitly defined as a~default identifier of a~node.

\(\no\) is implicitly defined as an~identifier of a~non-existing node (e.g.~if \(v\) is a~leaf, \(\tleft(v) = \n\)).

Dashed edges 
\begin{minipage}{1.5em}
\includesvg[width=\textwidth]{intro__dashed_edge}
\end{minipage}
depict edges that may or may not exist (e.g.~when we describe transformations and we do not care if the node is root or if it has children).

Green edges 
\begin{minipage}{1.5em}
\includesvg[width=\textwidth]{intro__green_edge}
\end{minipage}
depict edges to \(\no\) nodes.

Virtual node is an~artificially defined node that does not affect the structure definition (e.g.~we can define virtual leafs that are children of actual leaves of the binary tree. Such virtual leaves are not included when counting the number of children of the node, so the actual leaves are still called leaves).

Actual node is a~node that is not virtual.

We say that node \(v\) is an~virtual-oneling \(\textiff\) \(v\) has no siblings being virtual nodes.

We say that node \(v\) is virtual-twinling \(\textiff\) \(v\) has at least one sibling being virtual node.

For a~binary tree, we say that node \(v\) is right (left) \(\textiff\) \(v\) is a~right (left) child of its parent.

For a~binary tree, we say that node \(v\) has a~right (left) parent \(u\) \(\textiff\) \(v\) is a~left (right) child of \(u\).

\(\pointed{v}\) is implicitly defined as a~pointed node \(v\). We can treat it as a~node with an~additional label \(\pointed{}\).

\(v\weighted{n}\) is implicitly defined as a~weighted node \(v\) with weight \(n\).

\(x_i\) is implicitly assumed to be the~\(i\)-th element of a sequence \(X\), where \(x\) can be any lower-case letter and \(X\) is upper-case \(x\).

Calligraphic \(\gf{X}\) is implicitly assumed to be a~generating function of the sequence \(X\) where \(X\) is any upper-case letter.

\(\size{X}\) is implicitly defined as the size of the object \(X\).

\subsection{Structurally recursive objects}%
\label{sub:structurally_recursive_objects}

We will be analyzing structures that can be expressed using recursive formulas. For example we can define a~full binary tree recursively as a~node having no children (i.e.~being a~leaf) or having exactly two subtrees being full binary trees. To express the disjoint alternative in the construction rule, we use \(+\). The reason will become obvious soon.

\begin{figure}[H]
    \begin{center}
        \begin{minipage}[t]{.2\textwidth}
            \begin{center}
                \(B\)\\
                \includesvg[scale=0.5]{binary__def_1}%
            \end{center}
        \end{minipage}%
        \begin{minipage}[t]{.05\textwidth}
            \begin{center}
                \(=\)\\
            \end{center}
        \end{minipage}%
        \begin{minipage}[t]{.2\textwidth}
            \begin{center}
                \(\text{leaf}\)\\
                \includesvg[scale=0.5]{binary__def_2}%
            \end{center}
        \end{minipage}%
        \begin{minipage}[t]{.05\textwidth}
            \begin{center}
                \(+\)\\
            \end{center}
        \end{minipage}%
        \begin{minipage}[t]{.2\textwidth}
            \begin{center}
                \(B^2\)\\
                \includesvg[scale=0.5]{binary__def_3}%
            \end{center}
        \end{minipage}%
    \end{center}
    \caption{Binary tree recursive definition.}%
    \label{fig:binary_recursion}
\end{figure}

\subsection{Generating functions}%
\label{sub:generating_functions}
\todo{Source}
\todo{Make the section simpler}

Throughout the thesis we use ordinary generating functions to describe and count structures.

\begin{definition}
    The ordinary generating function of a sequence \(A = {(a_i)}_{i=0}^{\infty}\) is defined as the following formal power series:
\[\gf{A}(z) = \sum_{i=0}^{\infty} a_i z^i.\]
\end{definition}

Note that in order to compute the \(k\)-th element of the sequence, namely \(a_k\), we just need to find the~constant corresponding to indeterminate \(z^k\).

If we have a sequence counting some objects and a corresponding generating function, we say that this function counts the objects.

\begin{example}%
    \label{ex:bin_gf}
    Let's consider the recursive definition of binary trees shown in Figure~\ref{fig:binary_recursion}. We can create a~functional equation to count trees with exactly \(n\) internal nodes. To do so, we make internal nodes contribute to the generating function with \(z^1 = z\) and leaves with \(z^0 = 1\):

\[\gf{B}(z) = 1 + z\gf{B}^2(z).\]

In order to get some intuition on how it works underneath, let's take a~look at how we can find the number of full binary trees of size \(2\). We can expand the recurrence a~bit:

\[\begin{array}{rclr}
        \gf{B}(z) &=& 1 + z\gf{B}^2(z)\\
                  &=& 1 + z\bigg(1 + z\gf{B}^2(z)\bigg)\bigg(1 + z\gf{B}^2(z)\bigg)\\
                  &=& 1 + \mhl{z}\bigg(1 + \mhl{z}\Big(\mhl{1} + z\gf{B}^2(z)\Big)\Big(\mhl{1} + z\gf{B}^2(z)\Big)\bigg)\bigg(\mhl{1} + z\gf{B}^2(z)\bigg) &(*)\\
                  &=& 1 + z\bigg(1 + z\Big(1 + z\gf{B}^2(z)\Big)\Big(1 + z\gf{B}^2(z)\Big)\bigg)\bigg(1 + z\Big(1 + z\gf{B}^2(z)\Big)\Big(1 + z\gf{B}^2(z)\Big)\bigg)\\
                  &=& 1 + z + 2 z^2 + \bigg(2 \gf{B}^2(z) + {\Big(1 + z \gf{B}^2(z)\Big)}^2\bigg) z^3
\end{array}\]

It is enough, as the term \(\bigg(2 \gf{B}^2(z) + {\Big(1 + z \gf{B}^2(z)\Big)}^2\bigg) z^3\) will contribute only to \(z^k\) where \(k > 2\). Therefore, we can see that there are \(2\) trees with exactly \(2\) internal nodes, as multiplier next to \(z^2\) is \(2\).

To visualize even more, let's take a~look at the line marked with \((*)\). The highlighted terms describe following tree:

\begin{figure}[H]
    \begin{center}
        \includesvg[scale=.7]{intro__iilll}
    \end{center}
    \caption{Tree corresponding to highlighted terms in \((*)\).}%
    \label{fig:tree_corresponding_to_mhl}
\end{figure}

\end{example}

There exist multiple well-known methods and tools that allow us to work with generating functions without the need of digging into details. The tool we will be using extensively in this work is the Maple package called GFUN~\cite{gfun}. If you want to learn more about algorithms used in the package, we strongly encourage reading the referenced article.

\subsection{Holonomic recurrences}%
\label{sub:holonomic_recurrences}
\todo{Extend the section}

Since most of the time we will be working with holonomic recurrences, let's introduce them.

\begin{definition}[\cite{holotoolkit}]%
    \label{def:holo_1}
    An infinite sequence \(A = {(a_i)}_{i=0}^{\infty}\) is holonomic (aka. P-finite, P-recursive or D-recursive) \(\textiff\) there exists a~holonomic recurrence for it, i.e.

\[\exists_{r \in \N} \exists_{p_0, \ldots, p_r \in \poly} \forall_{n \in \N} \sum_{i=0}^r p_i(n)a_{n+i} = 0.\]
\end{definition}

Note that the sum can only take a~finite number of subsequent elements of the sequence \(A\) into the relation.

\begin{example}
    The sequence \(a_n = \frac{5n - 3}{3n + 5}\) is holonomic, because it satisfies
    \[(3n + 5)(5n + 2) a_n - (5n - 3)(3n + 8) a_{n+1} = 0.\]
\end{example}

\begin{example}
    Sequences \(a_n = \sqrt{n}\), \(b_n = n^n\) are \textbf{not} holonomic, i.e.~none of them satisfies a~linear recurrence equation with polynomial coefficients~\cite{nonholo}.
\end{example}

\begin{definition}[\cite{complexity}]%
    \label{def:holo_2}
    Equivalently to Definition~\ref{def:holo_1}, \(A = {(a_i)}_{i=0}^{\infty}\) is holonomic \(\textiff\) there exists a~generating function in the following form:
    \[\sum_{i=0}^{r} p_i(z) \gf{A}^{(i)}(z) = b(z)\]
    for some fixed \(r \in \N\); \(b, p_0, \ldots, p_r \in \poly\) not all identical to zero.
\end{definition}

The equivalence of these two definitions assures us that if we find an~generating function equation describing the desired objects satisfying Definition~\ref{def:holo_2}, we will be able to find holonomic recurrence mentioned in Definition~\ref{def:holo_1}~\cite{complexity}.

\begin{example}%
    \label{ex:gfun-rec}
    For generating function counting full binary trees defined in Example~\ref{ex:bin_gf}, i.e.:
    \[\gf{B}(z) = 1 + z\gf{B}^2(z)\]
    we can use GFUN~\cite{gfun} to immediately find a~holonomic recurrence of our interest:

    \begin{lstlisting}
> with(gfun):
> RootOf(B = B^2*z + 1, B);
             2
    RootOf(_Z  z - _Z + 1)

> algfuntoalgeq(%, B(z));
     2
    B  z - B + 1

> algeqtodiffeq(%, B(z));
                              2      /d      \
    1 + (-1 + 2 z) B(z) + (4 z  - z) |-- B(z)|
                                     \dz     /

> rec := diffeqtorec(%, B(z), B(n));
    rec := {(2 + 4 n) B(n) + (-2 - n) B(n + 1), B(0) = 1}
    \end{lstlisting}

    Our holonomic recurrence is then
    \[\left\{\begin{array}{rcl}
                b_0 &=& 1\\
                (2 + 4 n) b_n + (-2 - n) b_{n + 1} &=& 0
    \end{array}\right..\]

    The most interesting part for us is the relation among the finite number of consecutive elements of the sequence.
\end{example}

\subsection{Combinatorial interpretations}%
\label{sub:combinatorial_interpretations}

Combinatorial interpretation is just a~fancy name for something you probably know and use. For example if you know the binomial coefficient \(\binom{n}{k}\), I bet it was introduced to you as something like: ``let's assume you have \(n\) distinguishable balls. If you want to choose \(k\) from them, then you can do it in \(\binom{n}{k}\) ways'' and now, whenever you see the term \(\binom{n}{k}\), you interpret it as the number of ways to choose \(k\) objects from the set of size \(n\). This is the combinatorial interpretation of the binomial coefficient.

Holonomic recurrences are very useful if we want to count objects of a~given size. Of course, size can be defined in more complex ways, e.g.~we may want to find the number of binary trees with \(n\) internal nodes having exactly \(m\) nodes being left child of its parent. Unfortunately, there is no commonly known method to find their intuitive interpretation in the world of the objects they describe.

In the thesis we try to gather methods and results that may be helpful for anyone trying to find such an~interpretation, as we believe that it may give better insight into the nature of the structure and make it easier to study the structure's properties.

\subsection{Pre-order notation}%
\label{sub:pre_order_notation}

In the thesis, if not stated otherwise, we will be working on rooted ordered unlabeled \(n\)-ary trees. To understand the terms, let's define them:
\begin{itemize}
    \item rooted tree is a~tree with single node marked as root. This allows us to define hierarchy of nodes using such terms as parent and child;
    \item ordered trees introduce order on children from the leftmost to the rightmost;
    \item unlabeled trees have nodes that are not distinguishable by their label. We will introduce soon kinds of nodes, but you shouldn't treat them as labels, but rather way to describe the structure of the node that is already present (e.g.~we can call node with just left child an~\(l\) node, but it is not its label. If the right child was added to the \(l\) node, it would not be an~\(l\) node anymore).
\end{itemize}
Additionally we allow that \(n\)-ary node has some of its children unset (e.g.~we can define a~binary node with just left child set, but not right one).

In order to make notation easier and unambiguous, we introduce pre-order notation of an~\(n\)-ary tree.

\begin{definition}
    For single node \(v\), the pre-order notation is defined as \(\enc{v}\).
\end{definition}

Here, \(\enc{v}\) is defined as follows:
\begin{itemize}
    \item if \(v\) is an~existing node, \(\enc{v} = \n\);
    \item if \(v\) is a~non-existing node (i.e.~we describe non-existing child of some node), \(\enc{v} = \no\).
\end{itemize}

\begin{definition}
    Pre-order notation of \(n\)-ary tree \(T\), namely \(\enc{T}\), is defined recursively as:
    \[\enc{T} = \enc{\troot(T)} \prod_{i=1}^{n} \enc{i\text{-th subtree of } \troot(T)}\]
    i.e.~catenation of the identifier of the root of the tree with encoded its subtrees from left to right.
\end{definition}

\begin{figure}[H]
    \centering
    \includesvg[scale=.7]{intro__oononnn}
    \caption{Tree described in the pre-order notation as: \(\n \n \no \n \no \no \no\).}%
    \label{fig:oononnn}
\end{figure}

We use green edges 
\begin{minipage}{1.5em}
\includesvg[width=\textwidth]{intro__green_edge}
\end{minipage}
to depict edges to \(\no\) nodes.

You can note that the encoding of a~given tree is unique. Unfortunately, not every sequence encodes proper tree.

We can also define prefixes and suffixes of \(T\) in relation to a~node \(v\).

\begin{definition}
    \(\tSub(v)\) is defined as a~whole subtree rooted with node \(v\).
\end{definition}

\begin{definition}
    Let \(v\) be a~node described using \(k\)-th symbol of \(\enc{T}\). Notation of the prefix of \(T\) in relation to \(v\), namely \(\enc{\tPre(T, v)}\), is defined as a~prefix of \enc{T} of size \(k - 1\).
\end{definition}

\begin{definition}
    Let \(n\) be the length of the sequence \(\enc{T}\). Notation of suffix of \(T\) in relation to \(v\), namely \(\enc{\tSuf(T, v)}\), is defined as a~suffix of \enc{T} of size \(n - \size{\enc{\tPre(T, v)}} - \size{\enc{\tSub(v)}}\). Note that the suffix may be an ordered forest.\todo{add to the example}
\end{definition}

In other words, prefix of the tree \(T\) in relation to \(v\) describes a~part of the tree that appears before subtree rooted in \(v\) in pre-order notation. Suffix describes part of the tree that appears after it.

\begin{figure}[H]
    \centering
    \includesvg[scale=.7]{intro__presuf_custom}
    \caption{Visualization of prefixes, suffixes and subtrees}%
    \label{fig:presuf}
\end{figure}

Throughout thesis we define some operations using this notation. Valid pre-order notation of \(n\)-ary tree (note that not all sequences are valid pre-order notations) can be easily translated back into a~tree. In order to do so, we just take first symbol of the notation and create the root accordingly. Then for each of the \(n\) children of the root, in sequence, we decode it recursively as if it was whole tree. Each such operation consumes some prefix of the notation. At the end we should be left with no symbols in the sequence.

As \(T\) and \(\enc{T}\) describe the same structures, notation of trees \(T\) and its encoding \(\enc{T}\) will be used interchangeably.

\section{Known results}%
\label{sec:known_results}

In this section we show combinatorial interpretations of holonomic recurrences for binary trees, unary-binary trees and Schröder trees. As to our knowledge these are only tree classes that have commonly known interpretations. Also the thought process leading to these interpretations is not obvious.

Each of these interpretations has its unique character, like different selection of the size definition or different approach to arranging the terms of the holonomic recurrence.

These interpretations have been source of ideas that we were incorporating in order to make our partial interpretation for the case of lambda trees in unary de Bruijn notation.

\subsection{Binary trees}%
\label{sub:binary_trees}

Let's start with probably the most known interpretation, full binary trees~\cite{binary}.

We define the size of a~tree as the number of its internal nodes (i.e.~nodes that are not leaves). Our generating function is therefore a~solution of the following equation

\[\gf{B}(z) = 1 + z\gf{B}^2(z)\]

After a~quick transformation, we get the following holonomic recurrence:

\[\left\{\begin{array}{rcl}
            b_0 &=& 1\\
            (n + 2)b_{n + 1} &=& 2 (2n + 1)b_n
\end{array}\right.\]

This equation actually have \(2\) solutions, but we can eliminate one of them due to assumption that we are counting objects, i.e.~we are interested only in solutions that give us natural numbers.

Numbers defined as above are also known as Catalan numbers.

It is easy to see that a~tree of size \(n\) (i.e.~with \(n\) internal nodes) has \(n + 1\) leaves and therefore \(2n + 1\) nodes in total. Analogically the tree of size \(n + 1\) has \(n + 2\) leaves and \(2n + 3\) nodes in total. If you see some resemblance to our holonomic recurrence, you are perfectly right.

We are going to create a~bijection between the left and right side of the recurrence. We will be using the term \textit{bijective language} to talk about the recurrence using the objects it describes and their transformations.
\todo{rewrite}

\begin{interpretation}
Let's translate the recurrence to the bijective language.
\begin{itemize}
    \item \(b_n\) is simply the set of all full binary trees of size \(n\),
    \item \((n + 2) b_{n + 1}\) is the set of all full binary trees of size \(n + 1\) with one of the leaves pointed,
    \item \((2n + 1) b_{n}\) is the set of all full binary trees of size \(n\) with one of the nodes (either internal or a~leaf) pointed (i.e.~distinguished),
    \item \(2\) at the beginning of \(2 (2n + 1) b_{n}\) indicates that we will be using \(2\) copies of the set defined in the previous bullet.
\end{itemize}

At the moment the left side of the recurrence is related to trees of size \(n + 1\), the right, on the other hand, of size \(n\). Therefore, we need to apply some transformations on trees.

Let's define two local operations for tree \(T\) with a~single node \(v\) pointed: \(R_l(T, v)\) and \(R_r(T, v)\). The term \textit{local} means that the operation can operate on just some finite fragment of a~tree with predefined size (e.g.~node and its 2 children, but cannot traverse the whole tree).

\begin{definition}
    \(\enc{R_l(T, v)} = \enc{\tPre(T, v)} \n \pointed{\n} \no \no \enc{\tSub(v)} \enc{\tSuf(T, v)}\)
\end{definition}

\begin{figure}[H]
    \centering
    \begin{minipage}{.25\textwidth}\includesvg[scale=0.35]{binary__remy_base}\end{minipage}%
    \(\Rightarrow\)
    \begin{minipage}{.4\textwidth}\includesvg[scale=0.5]{binary__remy_left}\end{minipage}%
    \caption{Visualization of \(\enc{R_l(T, v)}\)}%
    \label{fig:remy_left}
\end{figure}

\begin{definition}
    \(\enc{R_r(T, v)} = \enc{\tPre(T, v)} \n \enc{\tSub(v)} \pointed{\n} \no \no \enc{\tSuf(T, v)} \)
\end{definition}

\begin{figure}[H]
    \centering
    \begin{minipage}{.25\textwidth}\includesvg[scale=0.35]{binary__remy_base}\end{minipage}%
    \(\Rightarrow\)
    \begin{minipage}{.4\textwidth}\includesvg[scale=0.5]{binary__remy_right}\end{minipage}%
    \caption{Visualization of \(\enc{R_r(T, v)}\)}%
    \label{fig:remy_right}
\end{figure}

We can now take the bijective interpretation of \((2n + 1) B(n)\), i.e.~trees of size \(n\) with a~single node pointed. For each of such trees, \(T\) we create new one by applying \(R_l\) and \(R_r\) to the pair \(T', v\), where \(T'\) is \(T\) with removed pointers and \(v\) is a~node that is pointed in \(T\). Therefore we generate \(2(2n + 1) B(n)\) trees of size \(n + 1\).
\todo{rewrite}

The only thing that's left for us to verify is if we generate all trees of size \(n + 1\) with pointed leaves. As from the recursion, we know that \(2 (2n + 1) B(n)\) is total number of trees with single leaf pointed, it is enough to check if we do not generate any duplicate using \(R_l\) and \(R_r\). It can happen in 2 ways:
\begin{enumerate}
    \item operation itself generates duplicates or
    \item 2 different operations generate the same tree.
\end{enumerate}

First case can also be split into 2 subcases. Without loss of generality we can analyze just one of the operations, let it be \(R_l\). The second case is symmetric.
\begin{itemize}
    \item for some nodes \(u \neq v\), \(R_l(T, u) = R_l(T, v)\). Let \(i_u, i_v\) be the indices of respectively \(\enc{u}\) and \(\enc{v}\) in \(\enc{T}\). Let \(\pointed{i}_u, \pointed{i}_v\) be the indices of pointed node respectively in \(\enc{R_l(T, u)}\) and \(\enc{R_l(T, v)}\). Due to the definition of \(R_l\), we get that \(\pointed{i}_u = i_u + 1\) and \(\pointed{i}_v = i_v + 1\), which implies that \(u = v\) what we assumed false.
    \item for some trees \(T \neq S\), \(R_l(T, u) = R_l(S, v)\). Let \(i_u, i_v\) be the indices of respectively \(\enc{u}\) in \(\enc{T}\) and \(\enc{v}\) in \(\enc{S}\). Let \(\pointed{i}_u, \pointed{i}_v\) be the indices of pointed node respectively in \(\enc{R_l(T, u)}\) and \(\enc{R_l(S, v)}\). If \(\pointed{i}_u \neq \pointed{i}_v\), this case is trivially false, so we can assume that \(\pointed{i}_u = \pointed{i}_v\). Therefore, again from the definition of \(R_l\), we have that \(i_u = i_v\). That implies that \(\enc{\tPre(T, u)} \enc{\tSub(u)} \enc{\tSuf(T, u)} = \enc{\tPre(S, v)} \enc{\tSub(v)} \enc{\tSuf(S, v)} \), which then sums up to \(\enc{T} = \enc{S}\), which we assumed false.
\end{itemize}

In the second case it is enough to notice that \(R_l\) generates trees with pointed nodes being a~left child, and \(R_r\) generates trees with pointed nodes being right child. That immediately provides us with the conclusion that we are generating no duplicates.

\qed%
\end{interpretation}

\subsection{Unary-binary trees}%
\label{sub:unary_binary_trees}

Let's think, what would have changed if we introduce unary nodes to the binary case~\cite{motzkin}.

\begin{definition}%
    \label{def:unary_binary_tree}
    An unary-binary tree is an~ordered tree in which any node can have either 0, 1 or 2 children.
\end{definition}

We do not allow nodes with gaps here, e.g.~a~binary node with just one child or unary node with no children.

We will define size as number of edges. Note that a~tree having \(n\) edges has exactly \(n + 1\) nodes.

We can easily derive the generating function:

\[\gf{U}(z) = 1 + z\gf{U}(z) + z^2 \gf{U}^2(z)\]

And with a~bit of Maple magic~\cite{gfun}, we immediately get the following recurrence:

\[(n + 2)u_{n} = (2n + 1)u_{n - 1} + 3(n - 1)u_{n - 2}\]

\[\left\{\begin{array}{rcl}
            u_0 &=& 1\\
            u_1 &=& 1\\
            (n + 2)u_{n} &=& (2n + 1)u_{n - 1} + 3(n - 1)u_{n - 2}
\end{array}\right.\]

Definition~\ref{def:unary_binary_tree} is precise, but introduces a~new kind of node that we would have to take care of. We can avoid it by using a~trick to make unary-binary trees recall binary trees. We do not change the structure, just the notation.

\begin{definition}%
    \label{def:unary_binary_2}
    A left-leaning binary tree is a~binary tree with the following constraint added: if a~node has a~right child, it also has a~left child. Of course a node with just a left child can appear in such tree.
\end{definition}

The bijection between unary-binary trees and left-leaning binary ones is very simple. We just represent unary nodes as nodes with just a~left child. The other cases translate immediately to themselves. Please note that this transformation preserves the size of the tree (i.e.~number of edges).

The next step is to make binary trees full. Let's define virtual leaves.

\begin{definition}
    Virtual node is an~artificially defined node that does not affect the structure definition (i.e.~they are not included when counting the number of children of the node).
\end{definition}

\begin{definition}
    Actual node is a~node that is not virtual.
\end{definition}

\begin{definition}
    A virtual leaf is a~virtual node added to the node in place of a~non-existing child.
\end{definition}

As we have added \(n + 2\) virtual nodes, we have added as many virtual edges.\todo{notes?}

\begin{figure}[H]
    \centering
    \includeinlinesvg{.2}{unary_binary__base}%
    \(\Rightarrow\)%
    \includeinlinesvg{.3}{unary_binary__full}
    \caption{Unary-binary tree to full-binary tree transformation}%
    \label{fig:unary_binary_transformation}
\end{figure}

Please note that \(n + 2\) matches the multiplier of \(u_n\) in holonomic equation.

We will be using \(U(n)\) to describe sets of trees corresponding to \(u_n\) from the holonomic recurrence. The set \((n + 2) U(n)\) can be interpreted as trees of size \(n\) with single virtual leaf pointed.

\begin{definition}
    We say that node \(v\) is an~virtual-oneling \(\textiff\) \(v\) has no virtual siblings. Note that the root is an oneling even if it has no parent.
\end{definition}

\begin{definition}
    We say that node \(v\) is virtual-twinling \(\textiff\) \(v\) has at least one virtual sibling.
\end{definition}

\begin{definition}
    If \(u\) is parent of \(v\), we say that sibling of \(u\) is an~uncle of \(v\).
\end{definition}

\begin{interpretation}
To present the interpretation in bijective language, we will have to introduce several types of the nodes.

\begin{enumerate}
    \item Vro --- virtual leaf being right virtual-oneling;
    \item Vrt --- virtual leaf being right virtual-twinling. We will divide this case into:
        \begin{enumerate}
            \item VrtUrv --- Vri which uncle is a~right virtual leaf;
            \item VrtUra --- Vri which uncle is a~right actual node;
            \item VrtUla --- Vri which uncle is a~left actual node.
        \end{enumerate}
    \item Vlt --- virtual leaf left virtual-twinling. Note that from definition~\ref{def:unary_binary_2}, left virtual leave cannot have a~sibling being an~actual node. We can divide also this case into:
        \begin{enumerate}
            \item VltUrv --- Vlt which uncle is a~right virtual leaf;
            \item VltUra --- Vlt which uncle is a~right actual node;
            \item VltUla --- Vlt which uncle is a~left actual node;
        \end{enumerate}
    \item A --- actual node.
\end{enumerate}

Note that it covers all possible cases of nodes and all the cases are mutually disjoint. There is no such thing as VrtUlv, VltUlv nor Vlo due to Definition~\ref{def:unary_binary_2}.

All we need to show are transformations from the smaller cases, i.e.~pointed trees from \(U(n-1)\) and \(U(n-2)\) to \(U(n)\).

Let's start from \(U(n-1)\). Its multiplier is \(2n + 1\). It suggests that we can point to any node, as in a~tree having \(n - 1\) actual edges, we have \(n\) actual and \(n + 1\) virtual nodes.

In the case of \(U(n-2)\), the multiplier is \(3(n-1)\), which suggests that we should perform \(3\) distinct types of transformations on the trees with a~pointed actual node.

In the table below we show transformations that are applied to trees from \(U(n-1)\) and \(U(n-2)\) in order to achieve trees from \(U(n)\). We mark pointed node with \(\pointed{}\). You can find the case name in the bottom left corners of cells.

\begin{center}
    \begin{longtable}{| c | c | c | c |}
        \hline

        \(U(n-2)\) &
        \(U(n-1)\) &
               &
        \(U(n)\)\\
        \hline

        &
        A \includesvg[scale=0.4]{unary_binary__1si} &
        \(\Rightarrow\)&
        Vro \includesvg[scale=0.4]{unary_binary__0fid} \\
        \hline

        &
        Vlt \includesvg[scale=0.4]{unary_binary__1fjg} &
        \(\Rightarrow\)&
        VrtUrv \includesvg[scale=0.4]{unary_binary__0fjdOgf} \\
        \hline

        A \includesvg[scale=0.4]{unary_binary__2i} &
        &
        \(\Rightarrow\)&
        VrtUra \includesvg[scale=0.4]{unary_binary__0fjdOdi} \\
        \hline

        &
        Vro \includesvg[scale=0.4]{unary_binary__1fid} &
        \(\Rightarrow\)&
        VrtUla \includesvg[scale=0.4]{unary_binary__0fjdOgi} \\
        \hline

        &
        Vrt \includesvg[scale=0.4]{unary_binary__1fjd} &
        \(\Rightarrow\)&
        VltUrv \includesvg[scale=0.4]{unary_binary__0fjgOdf} \\
        \hline

        A \includesvg[scale=0.4]{unary_binary__2i} &
        &
        \(\Rightarrow\)&
        VltUra \includesvg[scale=0.4]{unary_binary__0fjgOdi} \\
        \hline

        A \includesvg[scale=0.4]{unary_binary__2i} &
        &
        \(\Rightarrow\)&
        VltUla \includesvg[scale=0.4]{unary_binary__0fjgOgi} \\
        \hline
    \end{longtable}
\end{center}


As you can see, we have satisfied all of the cases of trees from \(U(n-2)\), \(U(n-1)\) and \(U(n)\) with no repetitions, which concludes the construction.

\qed%
\end{interpretation}

\subsection{Schröder trees}%
\label{sub:schröder_trees}

The next case we will show is the case of Schröder trees~\cite{schroder}.

\begin{definition}%
    \label{def:schroder_tree}
    A Schröder tree is a~tree with no unary nodes. Empty children are not allowed (e.g.~binary node with just one child).
\end{definition}

Simply speaking, nodes can have zero or strictly more than one child.

\begin{figure}[H]
    \centering
    \includesvg[scale=0.4]{schroder__example_1}
    \includesvg[scale=0.4]{schroder__example_2}
    \includesvg[scale=0.4]{schroder__example_3}
    \includesvg[scale=0.4]{schroder__example_4}
    \(\ldots\)
    \caption{Possible cases of Schröder tree nodes}%
    \label{fig:schroder_trees}
\end{figure}

This definition is precise, but is difficult to implement as a~structural recursion. Let's define binary structure that will allow us to encode all Schröder trees.

\begin{definition}%
    \label{def:well_weighted}
    A well-weighted full binary tree is a~weighted tree such that:
    \begin{itemize}
        \item leaves are unweighted;
        \item internal nodes can have weight either \(1\) or \(2\);
        \item if an~internal node has weight 2, then its right child is not a~leaf.
    \end{itemize}
\end{definition}

Let's represent node \(v\) with a~weight \(n\) as \(v\weighted{n}\).

We can easily define the bijection \(\Phi\) between models from Definition~\ref{def:schroder_tree} and~\ref{def:well_weighted} using pre-order notation:
\[\Phi(T) = \left\{\begin{array}{lcl}
            \n \no \no &:& T = \n \no \no \\
            \n\weighted{1}\; T_1\; T_2 &:& T = \n\; T_1\; T_2\\
            \n\weighted{2}\; T_1\; \Phi(\n\; T_2 \ldots T_n) &:& T = \n\; T_1\; T_2 \ldots T_n
\end{array}\right.\]
\todo{image}

Let's define the size of the tree as a~number of its leaves. As leaves are translated to leaves and internal nodes to internal nodes, the bijection \(\Phi\) is size-preserving.

The generating function equation is then:

\[\left\{\begin{array}{rcl}
            \gf{T}(z) &=& \gf{T}_1(z) + \gf{T}_2(z) + z\\
            \gf{T}_1(z) &=& \gf{T}^2(z)\\
            \gf{T}_2(z) &=& \gf{T}(z)(\gf{T}_1(z) + \gf{T}_2(z))
\end{array}\right.\]

\(\gf{T}_1\) and \(\gf{T}_1\) are generating functions of trees rooted in nodes of weight \(1\) and \(2\) respectively. \(\gf{T}\) is a generating function for the whole well-weighted full binary tree.

To be able to use Maple to generate the corresponding recurrence, we need to transform this system of equations to a~single one. We can eliminate \(\gf{T}_1(z)\) first.

\[\left\{\begin{array}{rcl}
            \gf{T}(z) &=& \gf{T}^2(z) + \gf{T}_2(z) + z\\
            \gf{T}_2(z) &=& \gf{T}(z)(\gf{T}^2(z) + \gf{T}_2(z))
\end{array}\right.\]

\[\left\{\begin{array}{rcl}
            \gf{T}(z) &=& \gf{T}^2(z) + \gf{T}_2(z) + z\\
            \gf{T}_2(z)(1 - \gf{T}(z)) &=& \gf{T}^3(z)
\end{array}\right.\]

As we cannot use division to get \(\gf{T}_2(z)\), let's multiply the first equation by \(1 - \gf{T}(z)\).

\[\gf{T}(z)(1 - \gf{T}(z)) = \gf{T}^2(z)(1 - \gf{T}(z)) + \gf{T}^3(z) + z(1 - \gf{T}(z))\]

It can be actually used now to generate the holonomic recurrence.

\begin{lstlisting}
> with(gfun):
> RootOf(T*(1 - T) = T^2*(1 - T) + T^3 + z*(1 - T),  T);
               2
    RootOf(2 _Z  + (-z - 1) _Z + z)

> algfuntoalgeq(%, T(z));
       2
    2 T  + (-z - 1) T + z

> algeqtodiffeq(%, T(z));
                               2            /d      \                   2
    {z - 1 + (-z + 3) T(z) + (z  - 6 z + 1) |-- T(z)|, T(0) = RootOf(2 _Z  - _Z)}
                                            \dz     /

> diffeqtorec(%, T(z), T(n));

    {(n - 1) T(n) + (-3 - 6 n) T(n + 1) + (n + 2) T(n + 2), ...}

> subs(n = n - 1, %);

    {(-2 + n) T(n - 1) + (3 - 6 n) T(n) + (n + 1) T(n + 1), ...}
\end{lstlisting}

We can also move \(t_n\) to the other side, i.e.:

\[\left\{\begin{array}{rcl}
            t_1 &=& 1\\
            t_2 &=& 1\\
            3 (2 n - 1) t_n &=& (n - 2) t_{n - 1} + (n + 1) t_{n + 1}
\end{array}\right.\]

This may seem counterintuitive, as by default we want to describe how to generate larger cases from the smaller ones, but it will make the interpretation simpler.

Let's translate the recurrence to the bijective language then.
\begin{itemize}
    \item \((2 n - 1) T(n)\) describes well-weighted binary trees of size \(n\) (i.e.~with \(n\) leaves) with some node pointed (internal or leaf),
    \item \((n + 1) T(n + 1)\) describes well-weighted binary trees of size \(n + 1\) with some leaf pointed,
    \item \((n - 2) T(n - 1)\) describes well-weighted binary trees of size \(n - 1\) with some internal node pointed.
\end{itemize}

We will show the bijection between two sides of the recursion. We will transform each tree from \((2 n - 1) T(n)\) in three ways in order to achieve all trees from \((n + 1) T(n + 1)\). We will get some extra ones which are not well-weighted, but we will deal with them in a~minute.

Let's define three transformations, \(L_1\), \(L_2\), and \(R_1\) that transform tree with a~pointed node \(s\) into bigger tree, as follows. We use dashed edges 
\begin{minipage}{1.5em}
\includesvg[width=\textwidth]{intro__dashed_edge}
\end{minipage}
to depict edges that may or may not exist (i.e.~when the node can have parent or child, but it does not have to). Whenever we use solid edge, we expect this edge to actually exist.

\begin{center}
    \includeinlinescaledsvg{.16}{.5}{schroder__lr_base}%
    \(\xRightarrow{L_1}\)%
    \includeinlinescaledsvg{.16}{.5}{schroder__l1}%
    \hspace{.1\textwidth}%
    \includeinlinescaledsvg{.16}{.5}{schroder__lr_base}%
    \(\xRightarrow{L_2}\)%
    \includeinlinescaledsvg{.16}{.5}{schroder__l2}%

    \includeinlinescaledsvg{.16}{.5}{schroder__lr_base}%
    \(\xRightarrow{R_1}\)%
    \includeinlinescaledsvg{.16}{.5}{schroder__r1}%
\end{center}

You can see that trees generated by \(L_1\) and \(R_1\) are always well-weighted. The problem appears when applying \(L_2\) when \(s\) is a~leaf.

When \(s\) is a~leaf, we will look at its parent. It can fall into one of following cases, where \(t'\) can be any tree and \(t''\) cannot be a~leaf.

\begin{center}
    \begin{minipage}[t]{.3\textwidth}
        \begin{center}
            \includesvg[scale=0.5]{schroder__case_a}\\
            Case 1.
        \end{center}
    \end{minipage}%
    \begin{minipage}[t]{.3\textwidth}
        \begin{center}
            \includesvg[scale=0.5]{schroder__case_b}\\
            Case 2.
        \end{center}
    \end{minipage}%
    \begin{minipage}[t]{.3\textwidth}
        \begin{center}
            \includesvg[scale=0.5]{schroder__case_c}\\
            Case 3.
        \end{center}
    \end{minipage}%
\end{center}

When applying \(L_2\) we achieve following cases respectively:

\begin{center}
    \begin{minipage}[t]{.3\textwidth}
        \begin{center}
            \includesvg[scale=0.45]{schroder__case_a_l}\\
            Case 1.
        \end{center}
    \end{minipage}%
    \begin{minipage}[t]{.3\textwidth}
        \begin{center}
            \includesvg[scale=0.45]{schroder__case_b_l}\\
            Case 2.
        \end{center}
    \end{minipage}%
    \begin{minipage}[t]{.3\textwidth}
        \begin{center}
            \includesvg[scale=0.45]{schroder__case_c_l}\\
            Case 3.
        \end{center}
    \end{minipage}%
\end{center}

In case 1.~and 2.~we can just swap the labels as follows:

\begin{center}
    \begin{minipage}[t]{.3\textwidth}
        \begin{center}
            \includesvg[scale=0.45]{schroder__case_a_l_fixed}\\
            Case 1.
        \end{center}
    \end{minipage}%
    \begin{minipage}[t]{.3\textwidth}
        \begin{center}
            \includesvg[scale=0.45]{schroder__case_b_l_fixed}\\
            Case 2.
        \end{center}
    \end{minipage}%
\end{center}

It is easy to show that it constructs a bijection between \((n + 1) T(n + 1)\) and subset of \((2 n - 1) T(n)\) with \(L_1\), \(L_2\) and \(R_1\) applied, not falling into case 3. We will show that in a~moment.

To achieve trees from \((n - 2) T(n - 1)\), we can simply take trees from case 3.~and replace whole subtree with subtree of \(t''\) with its root pointed. In this way, we will lose \(2\) leaves and obtain each tree from \((n - 2) T(n - 1)\):

\begin{center}
    \begin{minipage}[t]{.3\textwidth}
        \begin{center}
            \includesvg[scale=0.45]{schroder__case_c_l_fixed}\\
            Case 3.
        \end{center}
    \end{minipage}%
\end{center}

To see the bijection, we can just draw all possible cases:

\begin{center}
    \begin{longtable}{| c | c | c | c | c |}
        \hline
        &
        \(\Rightarrow\)&
        \(R_1\) &
        \(L_1\) &
        \(L_2\) \\
        \hline

        \includeinlinescaledsvg{.23}{.35}{schroder__proof__21} &
        \(\Rightarrow\)&
        \includeinlinescaledsvg{.23}{.35}{schroder__proof__23} &
        \includeinlinescaledsvg{.23}{.35}{schroder__proof__22} &
        \begin{minipage}{.23\textwidth}
            \begin{center}
                \includeinlinescaledsvg{1}{.35}{schroder__proof__24}
                \(\Downarrow\)
                \includeinlinescaledsvg{1}{.35}{schroder__proof__24b}
            \end{center}
        \end{minipage}
        \\
        \hline

        \includeinlinescaledsvg{.23}{.35}{schroder__proof__11} &
        \(\Rightarrow\)&
        \includeinlinescaledsvg{.23}{.35}{schroder__proof__13} &
        \includeinlinescaledsvg{.23}{.35}{schroder__proof__12} &
        \begin{minipage}{.23\textwidth}
            \begin{center}
                \includeinlinescaledsvg{1}{.35}{schroder__proof__14}
                \(\Downarrow\)
                \includeinlinescaledsvg{1}{.35}{schroder__proof__14b}
            \end{center}
        \end{minipage}
        \\
        \hline

        \includeinlinescaledsvg{.23}{.35}{schroder__proof__31} &
        \(\Rightarrow\)&
        \includeinlinescaledsvg{.23}{.35}{schroder__proof__33} &
        \includeinlinescaledsvg{.23}{.35}{schroder__proof__32} &
        \begin{minipage}{.23\textwidth}
            \begin{center}
                \includeinlinescaledsvg{1}{.35}{schroder__proof__34}
                \(\Downarrow\)
                \includeinlinescaledsvg{1}{.35}{schroder__proof__34b}
            \end{center}
        \end{minipage}
        \\
        \hline

        \includeinlinescaledsvg{.23}{.35}{schroder__proof__51} &
        \(\Rightarrow\)&
        \includeinlinescaledsvg{.23}{.35}{schroder__proof__53} &
        \includeinlinescaledsvg{.23}{.35}{schroder__proof__52} &
        \includeinlinescaledsvg{.23}{.35}{schroder__proof__54} \\
        \hline

        \includeinlinescaledsvg{.23}{.35}{schroder__proof__41} &
        \(\Rightarrow\)&
        \includeinlinescaledsvg{.23}{.35}{schroder__proof__43} &
        \includeinlinescaledsvg{.23}{.35}{schroder__proof__42} &
        \includeinlinescaledsvg{.23}{.35}{schroder__proof__44} \\
        \hline

        \includeinlinescaledsvg{.23}{.35}{schroder__proof__61} &
        \(\Rightarrow\)&
        \includeinlinescaledsvg{.23}{.35}{schroder__proof__63} &
        \includeinlinescaledsvg{.23}{.35}{schroder__proof__62} &
        \includeinlinescaledsvg{.23}{.35}{schroder__proof__64} \\
        \hline

        \includeinlinescaledsvg{.23}{.35}{schroder__proof__71} &
        \(\Rightarrow\)&
        \includeinlinescaledsvg{.23}{.35}{schroder__proof__73} &
        \includeinlinescaledsvg{.23}{.35}{schroder__proof__72} &
        \includeinlinescaledsvg{.23}{.35}{schroder__proof__74} \\
        \hline

        \includeinlinescaledsvg{.23}{.35}{schroder__proof__81} &
        \(\Rightarrow\)&
        \includeinlinescaledsvg{.23}{.35}{schroder__proof__83} &
        \includeinlinescaledsvg{.23}{.35}{schroder__proof__82} &
        \includeinlinescaledsvg{.23}{.35}{schroder__proof__84} \\
        \hline

    \end{longtable}
\end{center}

In the left column you can see all possibilities of selecting \(s\) from a~tree of size \(n\). In \(R_1\), \(L_1\) and \(L_2\) columns we show how they are then transformed. It covers all possible cases of selecting leaf from a~tree of size \(n + 1\) and selecting an~internal node from a~tree of size \(n - 1\).

\section{Approaches}%
\label{sec:approaches}

As you could see in the examples above, most of the solutions are based on the simple analysis of the number of leaves, internal nodes and some of their special types. In this section, we will gather some methods that may be helpful when trying to analyze the problem of interpreting combinatorially some holonomic recurrences.

\subsection{Defining the size carefully}%
\label{sub:defining_the_size_carefully}

First thing to be done is to define the size carefully. We have a few standard choices for specifying the size of trees:
\begin{itemize}
    \item counting all existing nodes,
    \item counting internal nodes,
    \item counting leaves,
    \item counting virtual leaves (i.e.~virtual nodes being children of actual leaves),
    \item counting edges.
\end{itemize}

It may seem counterintuitive at first that this changes anything crucial, but the definition changes the shape of the holonomic recurrence. Sometimes changing the definition can drastically reduce the number of cases we have to consider.

There are two most common symptoms indicating that we may want to choose a different method of counting the size:
\begin{itemize}
    \item we get empty classes of objects, e.g.~if we analyze the problem of full binary trees and define the size as the number of all existing nodes, we cannot generate a tree of even size, therefore every second class is empty;
    \item we want to join trees using a newly created node, which will become the root of the new object and our equations are getting complicated. For example if we define the size as the number of internal nodes, such catenation creates a~tree of size being the sum of sizes of the components increased by \(1\). This \(+1\) can drastically clutter the holonomic recurrence. If we had defined the size as the number of leaves, the size of the new object would be just a~simple sum of the sizes of the smaller trees.
\end{itemize}

\subsection{Tree binarization}%
\label{sub:tree_binarization}

If we describe some structure we may be tempted to use multiple kinds of nodes, e.g.~unary and binary nodes. Having multiple kinds of nodes may cause problems, both in creating holonomic recurrence and finding the interpretation, as we have to keep track of how they can transform.

As the most common trees are binary ones, it may be worth an~effort to find some structure based on binary trees that can represent the original structure (see Section~\ref{sub:unary_binary_trees}).

\subsection{Custom weights assignment}%
\label{sub:weights_variablization}

Most of the time we are operating on trees, we define the size as a count of some objects (nodes or edges). We can assign weights to these objects and make the size a weighted sum. The weights can be variable or we can assign some other constant values (e.g.~small prime numbers).

\todo{rewrite to unary-binary}
In Section~\ref{sub:lambda_terms} we will define trees of lambda terms. If we would assign following weights to nodes:

\[\begin{array}{rcl}
        a~&:& 7\\
        l &:& 5\\
        s &:& 3\\
        o &:& 2\\
\end{array}\]

we would get significantly smaller classes

\begin{longtable}{| c | c |}
    \hline

    Size & Trees
    \\\hline
    
    2 &
    \begin{minipage}{.9\textwidth}\centering
    \includeinlinescaledsvg{.19}{.5}{lambda__2357__lambda2357_002_000000}
    \end{minipage}
    \\\hline

    5 &
    \begin{minipage}{.9\textwidth}\centering
    \includeinlinescaledsvg{.19}{.5}{lambda__2357__lambda2357_005_000000}
    \end{minipage}
    \\\hline

    7 &
    \begin{minipage}{.9\textwidth}\centering
    \includeinlinescaledsvg{.19}{.5}{lambda__2357__lambda2357_007_000000}
    \end{minipage}
    \\\hline

    8 &
    \begin{minipage}{.9\textwidth}\centering
    \includeinlinescaledsvg{.19}{.5}{lambda__2357__lambda2357_008_000000}
    \end{minipage}
    \\\hline

    10 &
    \begin{minipage}{.9\textwidth}\centering
    \includeinlinescaledsvg{.19}{.5}{lambda__2357__lambda2357_010_000000}
    \end{minipage}
    \\\hline

    11 &
    \begin{minipage}{.9\textwidth}\centering
    \includeinlinescaledsvg{.19}{.5}{lambda__2357__lambda2357_011_000000}
    \includeinlinescaledsvg{.19}{.5}{lambda__2357__lambda2357_011_000001}
    \end{minipage}
    \\\hline

    12 &
    \begin{minipage}{.9\textwidth}\centering
    \includeinlinescaledsvg{.19}{.5}{lambda__2357__lambda2357_012_000000}
    \end{minipage}
    \\\hline

    14 &
    \begin{minipage}{.9\textwidth}\centering
    \includeinlinescaledsvg{.19}{.5}{lambda__2357__lambda2357_014_000000}
    \includeinlinescaledsvg{.19}{.5}{lambda__2357__lambda2357_014_000001}
    \includeinlinescaledsvg{.19}{.5}{lambda__2357__lambda2357_014_000002}
    \end{minipage}
    \\\hline

    15 &
    \begin{minipage}{.9\textwidth}\centering
    \includeinlinescaledsvg{.19}{.5}{lambda__2357__lambda2357_015_000000}
    \end{minipage}
    \\\hline

    16 &
    \begin{minipage}{.9\textwidth}\centering
    \includeinlinescaledsvg{.19}{.5}{lambda__2357__lambda2357_016_000000}
    \includeinlinescaledsvg{.19}{.5}{lambda__2357__lambda2357_016_000001}
    \includeinlinescaledsvg{.19}{.5}{lambda__2357__lambda2357_016_000002}
    \includeinlinescaledsvg{.19}{.5}{lambda__2357__lambda2357_016_000003}
    \end{minipage}
    \\\hline

    % 17 &
    % \begin{minipage}{.9\textwidth}\centering
    % \includeinlinescaledsvg{.19}{.5}{lambda__2357__lambda2357_017_000000}
    % \includeinlinescaledsvg{.19}{.5}{lambda__2357__lambda2357_017_000001}
    % \includeinlinescaledsvg{.19}{.5}{lambda__2357__lambda2357_017_000002}
    % \includeinlinescaledsvg{.19}{.5}{lambda__2357__lambda2357_017_000003}
    % \includeinlinescaledsvg{.19}{.5}{lambda__2357__lambda2357_017_000004}
    % \end{minipage}
    % \\\hline

    % 18 &
    % \begin{minipage}{.9\textwidth}\centering
    % \includeinlinescaledsvg{.19}{.5}{lambda__2357__lambda2357_018_000000}
    % \end{minipage}
    % \\\hline

    % 19 &
    % \begin{minipage}{.9\textwidth}\centering
    % \includeinlinescaledsvg{.19}{.5}{lambda__2357__lambda2357_019_000000}
    % \includeinlinescaledsvg{.19}{.5}{lambda__2357__lambda2357_019_000001}
    % \includeinlinescaledsvg{.19}{.5}{lambda__2357__lambda2357_019_000002}
    % \includeinlinescaledsvg{.19}{.5}{lambda__2357__lambda2357_019_000003}
    % \includeinlinescaledsvg{.19}{.5}{lambda__2357__lambda2357_019_000004}
    % \includeinlinescaledsvg{.19}{.5}{lambda__2357__lambda2357_019_000005}
    % \includeinlinescaledsvg{.19}{.5}{lambda__2357__lambda2357_019_000006}
    % \end{minipage}
    % \\\hline

    % 20 &
    % \begin{minipage}{.9\textwidth}\centering
    % \includeinlinescaledsvg{.19}{.5}{lambda__2357__lambda2357_020_000000}
    % \includeinlinescaledsvg{.19}{.5}{lambda__2357__lambda2357_020_000001}
    % \includeinlinescaledsvg{.19}{.5}{lambda__2357__lambda2357_020_000002}
    % \includeinlinescaledsvg{.19}{.5}{lambda__2357__lambda2357_020_000003}
    % \includeinlinescaledsvg{.19}{.5}{lambda__2357__lambda2357_020_000004}
    % \includeinlinescaledsvg{.19}{.5}{lambda__2357__lambda2357_020_000005}
    % \includeinlinescaledsvg{.19}{.5}{lambda__2357__lambda2357_020_000006}
    % \includeinlinescaledsvg{.19}{.5}{lambda__2357__lambda2357_020_000007}
    % \end{minipage}
    % \\\hline

    % 21 &
    % \begin{minipage}{.9\textwidth}\centering
    % \includeinlinescaledsvg{.19}{.5}{lambda__2357__lambda2357_021_000000}
    % \includeinlinescaledsvg{.19}{.5}{lambda__2357__lambda2357_021_000001}
    % \includeinlinescaledsvg{.19}{.5}{lambda__2357__lambda2357_021_000002}
    % \includeinlinescaledsvg{.19}{.5}{lambda__2357__lambda2357_021_000003}
    % \includeinlinescaledsvg{.19}{.5}{lambda__2357__lambda2357_021_000004}
    % \includeinlinescaledsvg{.19}{.5}{lambda__2357__lambda2357_021_000005}
    % \includeinlinescaledsvg{.19}{.5}{lambda__2357__lambda2357_021_000006}
    % \end{minipage}
    % \\\hline

    % 22 &
    % \begin{minipage}{.9\textwidth}\centering
    % \includeinlinescaledsvg{.19}{.5}{lambda__2357__lambda2357_022_000000}
    % \includeinlinescaledsvg{.19}{.5}{lambda__2357__lambda2357_022_000001}
    % \includeinlinescaledsvg{.19}{.5}{lambda__2357__lambda2357_022_000002}
    % \includeinlinescaledsvg{.19}{.5}{lambda__2357__lambda2357_022_000003}
    % \includeinlinescaledsvg{.19}{.5}{lambda__2357__lambda2357_022_000004}
    % \includeinlinescaledsvg{.19}{.5}{lambda__2357__lambda2357_022_000005}
    % \includeinlinescaledsvg{.19}{.5}{lambda__2357__lambda2357_022_000006}
    % \includeinlinescaledsvg{.19}{.5}{lambda__2357__lambda2357_022_000007}
    % \includeinlinescaledsvg{.19}{.5}{lambda__2357__lambda2357_022_000008}
    % \includeinlinescaledsvg{.19}{.5}{lambda__2357__lambda2357_022_000009}
    % \includeinlinescaledsvg{.19}{.5}{lambda__2357__lambda2357_022_000010}
    % \end{minipage}
    % \\\hline

    % 23 &
    % \begin{minipage}{.9\textwidth}\centering
    % \includeinlinescaledsvg{.19}{.5}{lambda__2357__lambda2357_023_000000}
    % \includeinlinescaledsvg{.19}{.5}{lambda__2357__lambda2357_023_000001}
    % \includeinlinescaledsvg{.19}{.5}{lambda__2357__lambda2357_023_000002}
    % \includeinlinescaledsvg{.19}{.5}{lambda__2357__lambda2357_023_000003}
    % \includeinlinescaledsvg{.19}{.5}{lambda__2357__lambda2357_023_000004}
    % \includeinlinescaledsvg{.19}{.5}{lambda__2357__lambda2357_023_000005}
    % \includeinlinescaledsvg{.19}{.5}{lambda__2357__lambda2357_023_000006}
    % \includeinlinescaledsvg{.19}{.5}{lambda__2357__lambda2357_023_000007}
    % \includeinlinescaledsvg{.19}{.5}{lambda__2357__lambda2357_023_000008}
    % \includeinlinescaledsvg{.19}{.5}{lambda__2357__lambda2357_023_000009}
    % \includeinlinescaledsvg{.19}{.5}{lambda__2357__lambda2357_023_000010}
    % \includeinlinescaledsvg{.19}{.5}{lambda__2357__lambda2357_023_000011}
    % \includeinlinescaledsvg{.19}{.5}{lambda__2357__lambda2357_023_000012}
    % \end{minipage}
    % \\\hline

    % 24 &
    % \begin{minipage}{.9\textwidth}\centering
    % \includeinlinescaledsvg{.19}{.5}{lambda__2357__lambda2357_024_000000}
    % \includeinlinescaledsvg{.19}{.5}{lambda__2357__lambda2357_024_000001}
    % \includeinlinescaledsvg{.19}{.5}{lambda__2357__lambda2357_024_000002}
    % \includeinlinescaledsvg{.19}{.5}{lambda__2357__lambda2357_024_000003}
    % \includeinlinescaledsvg{.19}{.5}{lambda__2357__lambda2357_024_000004}
    % \includeinlinescaledsvg{.19}{.5}{lambda__2357__lambda2357_024_000005}
    % \includeinlinescaledsvg{.19}{.5}{lambda__2357__lambda2357_024_000006}
    % \includeinlinescaledsvg{.19}{.5}{lambda__2357__lambda2357_024_000007}
    % \includeinlinescaledsvg{.19}{.5}{lambda__2357__lambda2357_024_000008}
    % \includeinlinescaledsvg{.19}{.5}{lambda__2357__lambda2357_024_000009}
    % \includeinlinescaledsvg{.19}{.5}{lambda__2357__lambda2357_024_000010}
    % \includeinlinescaledsvg{.19}{.5}{lambda__2357__lambda2357_024_000011}
    % \includeinlinescaledsvg{.19}{.5}{lambda__2357__lambda2357_024_000012}
    % \end{minipage}
    % \\\hline
\end{longtable}

This method gives us some suggestion which kind of nodes are modified in the operation

Another benefit of this method is decreasing the cardinalities of sets of objects of a given size, so we have less objects to look at.

Unfortunately, decreasing the cardinalities of the classes increases the number of sets we need to analyze to get a~full interpretation.

\subsection{Inductive translation}%
\label{sub:inductive_translation}

Another method that may help us finding the interpretation is inductive translation~\cite{doron}. The method is based on the idea that most of the time there is an~easy way to find some natural recurrence (not necessarily holonomic) that has simple interpretation in bijective language.

In the case of binary trees with size defined as the number of leaves, we can define 2 recurrences:

\[\text{Holonomic recurrence} : n b_{n} = 2 (2n - 3) b_{n - 1}\]
\[\text{Non-linear recurrence} : b_{n} = \sum_{i=1}^{n - 1} b_{i} b_{n - i}\]

The non-linear one has a~very simple interpretation. We can take two smaller trees and join them using a newly created common root.

We can multiply both sides of non-linear recurrence by \(n\) and break up \(n = i + (n - i)\) to get two similar sums

\[n b_{n} = \sum_{i=1}^{n - 1} i b_{i} b_{n - i} + \sum_{i=1}^{n - 1} (n - i) b_{n - i} b_{i}\]

Using our inductive hypothesis, i.e.~holonomic recurrence and moving problematic boundary cases out from the sums (in this case \(b_{1}\) needs to be analyzed separately, as we did not define binary trees with no leaves), we can get the right side to use \(b_{n - 1}\) and prove that the holonomic recurrence using non-linear one. The only part that is left is to translate this inductive proof to the bijective language, which in such a~simple case as binary trees is quite simple, especially if one has seen the solution once.

The method itself has proven to be helpful in order to divide the problem into smaller parts and find some corner cases. Unfortunately, it does not give any immediate interpretation for applying the inductive step.
\todo{fill the gaps}
\todo{add reconstruction images}

\section{Brutal iterative approach}%
\label{sec:brutal_iterative_approach}

If all manual methods have failed and the problem is still too complex to interpret it by hand, a~computer comes in handy.

As our inspiration and main field of interest is an analysis of trees of lambda terms in unary de Bruijn notation, we will introduce them and use them as an~example to describe the method.

\subsection{Lambda terms}%
\label{sub:lambda_terms}

Lambda calculus is a formal system that allows expressing computable functions introduced by Alonzo Church. It is widely used in computer science and mathematics to study functions and their properties.

Let a countable set of variables be given.

\begin{definition}
    Set of lambda terms \(\Lambda\) is defined recursively:
    \begin{itemize}
        \item if \(x\) is a~variable, then \(x \in \Lambda\)
        \item if \(M, N \in \Lambda\), then \((M N) \in \Lambda\) (it can be interpreted as an application of function \(M\) to an~argument \(N\))
        \item if \(x\) is a~variable and \(M \in \Lambda\), then \((\lambda x. M) \in \Lambda\) (it can be interpreted as an~abstraction, i.e.~creating a~function with an~argument \(x\) and binding each occurrence of \(x\) in \(M\) to this argument. We say that variable \(x\) is abstracted by this lambda-abstraction)
    \end{itemize}
\end{definition}

This definition resembles tree-based structure of lambda terms, where:
\begin{itemize}
    \item lambda abstraction is represented as an unary node holding name of lambda-abstracted variable;
    \item application is represented as a binary node;
    \item variable is represented as a leaf holding name of the variable.
\end{itemize}

\begin{figure}[H]
    \centering
    \includesvg[scale=.7]{lambda__tree_structure__example_custom}
    \caption{Tree representation of the term \(\lambda x. (\lambda y.x) (\lambda y.y)\).}%
    \label{fig:lambda_tree_example}
\end{figure}

We have used dashed arrows to point lambda-abstraction binding particular variables. It is also possible that a variable is unbound, i.e.~there is no lambda-abstraction that abstracts the variable. We are perfectly fine with that.

The definition using named variables is quite popular, but introduces several problems. One of them is a~problem of a potential collision of lambda-abstracted variable names and difficulty of finding to which lambda abstraction the variable is bound (or is it even a~free variable, i.e.~not bound). To address that, we can use de Bruijn indices.

\begin{definition}
    Set of lambda terms \(\LambdadB\) is defined recursively:
    \begin{itemize}
        \item if \(n \in \N\), then \(n \in \LambdadB\) (\(n\) is a~variable bound to lambda abstraction \(n\) levels above, i.e.~if you traverse the tree of the lambda term upwards, we have to ignore \(n\) lambda-abstraction nodes and the \((n+1)\)-st one is the one binding the variable. If we would have to go beyond the root of the tree, we say that the variable is free, i.e.~not bound)
        \item if \(M, N \in \LambdadB\), then \((M N) \in \LambdadB\) (application)
        \item if \(M \in \LambdadB\), then \((\lambda M) \in \LambdadB\) (abstraction)
    \end{itemize}
\end{definition}

As you can see, we managed to eliminate name-based variables. We can also easily check if a~given variable is bound by checking its depth.

\begin{example}
    \(\lambda x.\lambda y.x\) is equivalent to \(\lambda \lambda 1\)
\end{example}

\begin{example}
    \(\lambda x. (\lambda y.x) (\lambda y.y)\) is equivalent to \(\lambda (\lambda 1) (\lambda 0)\)
\end{example}

We can define tree representation of the terms from \(\LambdadB\). In order to encode natural numbers, we use unary notation. This allows us to connect the size of the tree to de Bruijn index values.

\begin{definition}
    Translation from lambda terms in de Bruijn notation to binary trees \(\Phi\) is defined as follows:
    \begin{itemize}
        \item \(\Phi(0) = \n\; \no\; \no\)
        \item \(\Phi(n) = \n\; \no\; \Phi(n - 1)\) (this node acts as a successor in unary notation, i.e.~function incrementing natural number by \(1\))
        \item \(\Phi(\lambda M) = \n\; \Phi(M)\; \no\)
        \item \(\Phi(M N) = \n\; \Phi(M)\; \Phi(N)\)
    \end{itemize}
\end{definition}

\begin{figure}[H]
    \centering
    \includesvg[scale=.7]{lambda__tree_structure__deBruijn_001_custom}
    \caption{Tree representation of the term \(\lambda (\lambda 1) (\lambda 0)\).}%
    \label{fig:lambda_tree_example_2}
\end{figure}

We use green edges to express the non-existing child of the node. Note that in order to get the exact value of the variable we had to take a look at a leaf with all successor nodes above it.

To simplify the interpretation, we can introduce several kinds of the nodes: \(o\), \(s\), \(l\) and \(a\). They do not change the structure, but make the reading a~bit easier:

\begin{definition}
    \(o\) is a~leaf.
\end{definition}

\begin{definition}
    \(s\) is a~node with only a right child.
\end{definition}

\begin{definition}
    \(l\) is a~node with only a left child.
\end{definition}

\begin{definition}
    \(a\) is a~node with both children.
\end{definition}

This leads to the following interpretation of \(\Phi\):

\begin{itemize}
    \item \(\Phi(0) = o\; \no\; \no\)
    \item \(\Phi(n) = s\; \no\; \Phi(n - 1)\)
    \item \(\Phi(\lambda M) = l\; \Phi(M)\; \no\)
    \item \(\Phi(M N) = a\; \Phi(M)\; \Phi(N)\)
\end{itemize}

\begin{figure}[H]
    \centering
    \includesvg[scale=.7]{lambda__tree_structure__deBruijn_002_custom}
    \caption{Tree representation of the term \(\lambda (\lambda 1) (\lambda 0)\) with node kinds.}%
    \label{fig:lambda_tree_example_3}
\end{figure}

You can note that every tree has its own binary tree representation, but not every binary tree represents a~valid term.

\begin{example}
    \(\n\; \no\; \n\; \n\; \no\; \no\; \no\) does represent a~valid tree, but not a~valid lambda term.

To see that, we can take a~look at this tree with our node kinds used:

\begin{figure}[H]
    \centering
    \includesvg[scale=.7]{lambda__tree_structure__deBruijn_invalid}
    \caption{Visualization of the tree \(\n\; \no\; \n\; \n\; \no\; \no\; \no\) with node kinds.}%
    \label{fig:lambda_tree_invalid}
\end{figure}

As \(s\) represents the successor, it can be applied only to \(n \in \N\), but is applied to the lambda-abstracted term.
\end{example}

\subsection{Enhanced holonomic recurrence}%
\label{sub:enhanced_holonomic_reccurence}

As we have our structure defined, let's create a generating function that will keep track of more than just the number of nodes, but also counts the number of each kind of node in the tree. The generating function \(\gf{T}(z, o, l, a, s)\) will be parameterized then using the following arguments:
\begin{itemize}
    \item \(z\) marking all actual nodes
    \item \(o\) marking variable nodes
    \item \(l\) marking lambda abstraction nodes
    \item \(a\) marking application nodes
    \item \(s\) marking successors of variable nodes
\end{itemize}

In order to make the image cleaner, we will drop the arguments of generating functions, i.e.~\(\gf{T} := \gf{T}(z, o, l, a, s)\).


\begin{center}
    \begin{minipage}[t]{.2\textwidth}
        \begin{center}
            \(\gf{T}\)\\
            \includesvg[scale=0.5]{lambda__def__1}%
        \end{center}
    \end{minipage}%
    \begin{minipage}[t]{.05\textwidth}
        \begin{center}
            \(=\)\\
        \end{center}
    \end{minipage}%
    \begin{minipage}[t]{.2\textwidth}
        \begin{center}
            \(z a~\gf{T}^2\)\\
            \includesvg[scale=0.5]{lambda__def__2}%
        \end{center}
    \end{minipage}%
    \begin{minipage}[t]{.05\textwidth}
        \begin{center}
            \(+\)\\
        \end{center}
    \end{minipage}%
    \begin{minipage}[t]{.2\textwidth}
        \begin{center}
            \(z l \gf{T}\)\\
            \includesvg[scale=0.5]{lambda__def__3}%
        \end{center}
    \end{minipage}%
    \begin{minipage}[t]{.05\textwidth}
        \begin{center}
            \(+\)\\
        \end{center}
    \end{minipage}%
    \begin{minipage}[t]{.2\textwidth}
        \begin{center}
            \(\gf{S}\)\\
            \includesvg[scale=0.5]{lambda__def__4}%
        \end{center}
    \end{minipage}%
\end{center}

where \(V\) represents part of the tree being variable (i.e.~\(o\) node, potentially with some \(s\) successor nodes applied):

\begin{center}
    \begin{minipage}[t]{.2\textwidth}
        \begin{center}
            \(\gf{V}\)\\
            \includesvg[scale=0.5]{lambda__def__5}%
        \end{center}
    \end{minipage}%
    \begin{minipage}[t]{.05\textwidth}
        \begin{center}
            \(=\)\\
        \end{center}
    \end{minipage}%
    \begin{minipage}[t]{.2\textwidth}
        \begin{center}
            \(z s \gf{V}\)\\
            \includesvg[scale=0.5]{lambda__def__4}%
        \end{center}
    \end{minipage}%
    \begin{minipage}[t]{.05\textwidth}
        \begin{center}
            \(+\)\\
        \end{center}
    \end{minipage}%
    \begin{minipage}[t]{.2\textwidth}
        \begin{center}
            \(z o\)\\
            \includesvg[scale=0.5]{lambda__def__6}%
        \end{center}
    \end{minipage}%
\end{center}

After solving the equation for \(\gf{V}\), we get:
\[\gf{T} = z a~\gf{T}^2 + z l \gf{T} + \frac{z o}{1 - z s}\]

If we treat \(z\) as the only variable and make \(o\), \(l\), \(a\), and \(s\) constants, we can use a~little bit of Maple magic~\cite{gfun} to achieve the following holonomic recurrence.

\[\begin{array}{rl}
        0 =& (n - 4) l^2 s^2 t_{n - 4}\\
        +& ((4 n - 10) a o s - (2 n - 6) l^2 s - (2 n - 5) l s^2) t_{n - 3}\\
        +& ((-4 n + 8) a o + (n - 2) l^2 + (4 n - 6) l s + (n - 1) s^2) t_{n - 2}\\
        +& ((-2 n + 1) l - 2 n s) t_{n - 1}\\
        +& (n + 1) t_{n}
\end{array}.\]

Terms \(o\), \(l\), \(a\), and \(s\) appearing in the holonomic recurrence can be interpreted as list of kinds of nodes we need to use to extend smaller tree into bigger one.

\begin{example}
    Let's consider one of the summands, \((2n - 5) l s^2 t_{n - 3}\). We can interpret it as extending trees of size \((n - 3)\). In total we want to have \((2n - 5) t_{n - 3}\) of them and each of them should have some node pointed (we allow duplicates). To extend the tree we need to use exactly one \(l\) node and two \(s\) nodes.
\end{example}

If we forget about kinds of nodes, we get following recurrence:

\[\begin{array}{rl}
        0 =& (n - 4) t_{n - 4}\\
        +& t_{n - 3}\\
        +& (2 n - 1) t_{n - 2}\\
        +& (-4 n + 1) t_{n - 1}\\
        +& (1 + n) t_{n}
\end{array}.\]

You may notice that some summands have become oversimplified. For example \(((4 n - 10) a o s - (2 n - 6) l^2 s - (2 n - 5) l s^2) t_{n - 3}\) has become \(t_{n-3}\), which could be interpreted as simply set of trees of size \(n - 3\) without any pointing.

\subsection{Transformator framework}%
\label{sub:the_framework}

We also created a~framework~\cite{transformator}, called Transformator, that may help visualize our work. It operates on binary trees and is based on the idea that every transformation expressed in bijective language is local, i.e.~transforms some node and a finite number of its surrounding nodes.

To show the idea of such transformations, let's define transformation context with holes.

\begin{definition}
    The transformation context is a rooted tree. Some of nodes can be left as holes, that may be later filled with trees. It also specifies type of the edge above the root of the context (i.e.~if that is actual edge, optional actual edge, virtual edge, or if there is no edge at all).
\end{definition}

\begin{example}%
    \label{ex:transformation}%
    We want to modify a node being right a child of its parent. We do not care what is below that node (i.e.~if that is a leaf or if it has children). We want to add a new node of which the original one is left child. We do not want the new node to have a right child. Additionally we want to swap potential children of the original node. We can define following transformation:
    \begin{center}
        \includeinlinescaledsvg{.4}{.7}{lambda__contexts__def_001}%
        \(\Rightarrow\)
        \includeinlinescaledsvg{.4}{.7}{lambda__contexts__def_002}%
    \end{center}
\end{example}

\(H_1\), \(H_2\), and \(H_3\) are holes. Edge crossing the box around the context defines type of edge the context is connect to the rest of the tree (i.e.~optional actual edge). We marked the node being described to clarify the picture.

We support following starting contexts of the transformations:

\begin{itemize}
    \item node is a~virtual leaf (each tree with \(n\) actual nodes has exactly \((n + 1)\) places where such contex apply):\\
        \includeinlinescaledsvg{1}{.7}{lambda__contexts__type_001}%
    \item node is an~actual node (tree with \(n\) actual nodes has excatly \(n\) places where such context apply):\\
        \includeinlinescaledsvg{1}{.7}{lambda__contexts__type_002}%
    \item node is a~root (each tree has exactly \(1\) place where such context apply):\\
        \includeinlinescaledsvg{1}{.7}{lambda__contexts__type_003}%
    \item node has its parent on the left, i.e.~is right child of its parent (the node can be specified to be actual or virtual):\\
        \includeinlinescaledsvg{.5}{.7}{lambda__contexts__type_004}%
        \includeinlinescaledsvg{.5}{.7}{lambda__contexts__type_004b}%
    \item node has its parent on the right, i.e.~is the left child of its parent (the node can be specified to be actual or virtual):\\
        \includeinlinescaledsvg{.5}{.7}{lambda__contexts__type_005}%
        \includeinlinescaledsvg{.5}{.7}{lambda__contexts__type_005b}%
\end{itemize}

The latest two are introduced in order to make the analysis simpler, but do not have immediate interpretation in the world of holonomic recurrences. However, if we use them both, it sums up to all nodes except root (as root is neither left nor right child), so in total there are \((n - 1)\) places where such a pair of contexts apply.

The target of the transformation can be any context preserving all holes. This allows us to specify number of nodes added by the transformation.

\subsection{Transformation classes}%
\label{sub:transformation_classes}

Note that for each transformation we can specify number and kinds of the nodes added to the tree. Let's take a look at the transformation from Example~\ref{ex:transformation}:
\begin{center}
    \includeinlinescaledsvg{.4}{.7}{lambda__contexts__def_001}%
    \(\Rightarrow\)
    \includeinlinescaledsvg{.4}{.7}{lambda__contexts__def_002_with_l}%
\end{center}
It adds single \(l\) node.

We will be operating on transformation classes, sections. They will start being empty and we will be adding tranformations to them.

\(\tSection_\tau\), where \(\tau\) is list constructed from \(\{a, l, o, s\}\) (you can treat is as a~set, but with potential repetitions), stores transformation extending the tree with nodes \(\tau\).

\(\tClass_i\), where \(i \in \N\) is defined as \(\bigcup_{\tau : |\tau| = i} \tSection_\tau\) (i.e.~set of all transformations adding exactly \(i\)~nodes to the tree).

We will be also talking about trees generated by the transformation. Let's consider constant \(n \in \N\). List of trees generated by the transformation from \(\tSection_\tau\) is defined as all trees generated by applying the transformation to all trees of size \((n - |\tau|)\).

Analogically, we can specify list of trees generated by the \(\tSection_\tau\) and \(\tClass_i\) as list of trees generated by all transformations from them.

Each section is associated with an expression coming from the holonomic recurrence. It defines expected number of trees that will be generated by each of them. Let \(k = n - |\tau|\) for each \(\tSection_\tau\). In the case of trees of lambda terms, we get following expressions associated to each section:

\[\begin{array}{lcr}
        \tSection_{\emptyset} &:& (k + 1) t_k\\
        \tSection_{\{l\}} &:& - (2 k + 1) t_k\\
        \tSection_{\{s\}} &:& - 2 (k + 1) t_k\\
        \tSection_{\{ao\}} &:& - (4 k) t_k\\
        \tSection_{\{ll\}} &:& (k) t_k\\
        \tSection_{\{ls\}} &:& (4 k + 2) t_k\\
        \tSection_{\{ss\}} &:& (k + 1) t_k\\
        \tSection_{\{aos\}} &:& 2 (2 k + 1) t_k\\
        \tSection_{\{lls\}} &:& - (2 k) t_k\\
        \tSection_{\{lss\}} &:& - (2 k + 1) t_k\\
        \tSection_{\{llss\}} &:& (k) t_k\\
\end{array}\]

Using \(k\) instead of \(n\) makes it easier to interpret, as we immediately see number of trees that should be generated by each section. For example \(\tSection_{\{ll\}}\) should generate \(k t_k\) trees from trees of size \(k\), which can be interpreted as taking each actual node and adding two \(l\) nodes above it.

Note that some of these expressions will evaluate to negative numbers. It is because of the fact that sum of all sections is equal to zero. We will be generating two kinds of trees then: positive (contributing to the total number of trees being generated as \(1\)) and negative ones (contributing as \(-1\)).

\subsection{Iteration}%
\label{sub:iteration}

As we want to make our work kind of organized, we are going to divide the process of finding interpretation into iterations. Each iteration should be a~small step towards solving the original problem.

Our brains like small examples, so we will consider small constant \(n = 3\). It will be describing size of the trees we want to generate. Because we start with such small \(n\), some of expressions associated with some sections will be empty, but it is ok. That's less trees to look at and soon we will increase \(n\).

We will be still using \(k = n - |\tau|\) when considering sections to make the view simpler.

Let's take a look at the number of trees we should generate:

\begin{lstlisting}
=== Classes stats ===
[ ]  0,0:       0 of      16 | (k + 1) * Tk              []
[ ]  1,0:       0 of     -10 | - (2 * k + 1) * Tk        [l]
[ ]  1,1:       0 of     -12 | - 2 * (k + 1) * Tk        [s]
[ ]  2,0:       0 of      -4 | - (4 * k) * Tk            [ao]
[ ]  2,1:       0 of       1 | (k) * Tk                  [ll]
[ ]  2,2:       0 of       6 | (4 * k + 2) * Tk          [ls]
[ ]  2,3:       0 of       2 | (k + 1) * Tk              [ss]
[ ]  3,0:       0 of       2 | 2 * (2 * k + 1) * Tk      [aos]
[x]  3,1:       0 of       0 | - (2 * k) * Tk            [lls]
[ ]  3,2:       0 of      -1 | - (2 * k + 1) * Tk        [lss]
[x]  4,0:       0 of       0 | k * Tk                    [llss]
\end{lstlisting}

The listing above is statistics view from Transformator framework. Let's analyze the first line:
\begin{itemize}
    \item \verb|[ ]| at the beginning means that we still need to generate some more trees. If we have generated all of them, it would be \verb|[x]|;
    \item \verb|0,0| is class and section identifier, i.e. \(|\tau|\) and arbitrarily assigned section identifier;
    \item \verb|0 of 16| means that we have generated \(0\) trees from \(16\) expected;
    \item \verb|(k + 1) * Tk| is just the expression associated with the section;
    \item \verb|[]| at the end is \(\tau\), which in this case is equal to \(\emptyset\).
\end{itemize}

We start with interpreting \(\tSection_\emptyset\). From statistics above, we see that we should generate \(16\) trees. \((k + 1)\) suggests pointing to a~virtual leaf. Let's do so. We do that by defining following transformation:

\begin{center}
    \includeinlinescaledsvg{.4}{.7}{lambda__transformations__001a}%
    \(\xRightarrow{+1}\)%
    \includeinlinescaledsvg{.4}{.7}{lambda__transformations__001b}%
\end{center}

Here \(\xRightarrow{+1}\) means that we are generating positive trees.

In the language of Transformator framework it would be:

\begin{lstlisting}
c.define_class(0)

c.define_section(0)

c.append_r_class_subtree_visitor(1,       # we define positive trees
                                          # i.e. that count as 1
    lambda kind, left, right:
        None if kind[0] is not None else  # we ignore all actual nodes
        [kind + ("*", )] + left + right   # for each virtual node we point it
)
\end{lstlisting}

After such operation we get following statistics:

\begin{lstlisting}
# define 0,0: (k + 1) * Tk []
=== Classes stats ===
[x]  0,0:      16 of      16 | (k + 1) * Tk              []
[ ]  1,0:       0 of     -10 | - (2 * k + 1) * Tk        [l]
[ ]  1,1:       0 of     -12 | - 2 * (k + 1) * Tk        [s]
[ ]  2,0:       0 of      -4 | - (4 * k) * Tk            [ao]
[ ]  2,1:       0 of       1 | (k) * Tk                  [ll]
[ ]  2,2:       0 of       6 | (4 * k + 2) * Tk          [ls]
[ ]  2,3:       0 of       2 | (k + 1) * Tk              [ss]
[ ]  3,0:       0 of       2 | 2 * (2 * k + 1) * Tk      [aos]
[x]  3,1:       0 of       0 | - (2 * k) * Tk            [lls]
[ ]  3,2:       0 of      -1 | - (2 * k + 1) * Tk        [lss]
[x]  4,0:       0 of       0 | k * Tk                    [llss]
=== Diff stats ===
0,0:     0    16
\end{lstlisting}

Note that we have generated all expected trees associated with \(\tSection_\emptyset\) (it is indicated by \verb|[x]|).

We also get the \verb|Diff stats|. It shows us information about how we have unbalanced the positive and negative side of our equation. At the end we want to generate the same list of trees at the negative and positive site.

The framework also can provide us with the visualization of the trees we have generated. For example, from the tree on the left side of \(\xRightarrow{+1}\) we have generated four trees:

\includeinlinesvg{.19}{lambda__trees_00__4_base}%
\(\xRightarrow{+1}\)%
\includeinlinesvg{.19}{lambda__trees_00__4}%
\includeinlinesvg{.19}{lambda__trees_00__5}%
\includeinlinesvg{.19}{lambda__trees_00__6}%
\includeinlinesvg{.19}{lambda__trees_00__7}%

You can see that there are four virtual leaves matching left side of the transformation we have defined.

We can also query Transformator framework to visualize the trees that are unbalanced. For \(\tClass_0\) those are:

(1)%
\begin{minipage}{.98\textwidth}\begin{center}%
\includeinlinescaledsvg{.33}{.4}{lambda__trees_00__2}%
\includeinlinescaledsvg{.33}{.4}{lambda__trees_00__3}%
\includeinlinescaledsvg{.33}{.4}{lambda__trees_00__7}%
\end{center}\end{minipage}

(2)%
\begin{minipage}{.98\textwidth}\begin{center}%
\includeinlinescaledsvg{.33}{.4}{lambda__trees_00__4}%
\includeinlinescaledsvg{.33}{.4}{lambda__trees_00__12}%
\includeinlinescaledsvg{.33}{.4}{lambda__trees_00__13}%
\end{center}\end{minipage}

(3)%
\begin{minipage}{.98\textwidth}\begin{center}%
\includeinlinescaledsvg{.16}{.4}{lambda__trees_00__0}%
\includeinlinescaledsvg{.16}{.4}{lambda__trees_00__1}%
\includeinlinescaledsvg{.16}{.4}{lambda__trees_00__5}%
\includeinlinescaledsvg{.16}{.4}{lambda__trees_00__6}%
\includeinlinescaledsvg{.16}{.4}{lambda__trees_00__14}%
\includeinlinescaledsvg{.16}{.4}{lambda__trees_00__15}%
\end{center}\end{minipage}

(4)%
\begin{minipage}{.98\textwidth}\begin{center}%
\includeinlinescaledsvg{.25}{.4}{lambda__trees_00__8}%
\includeinlinescaledsvg{.25}{.4}{lambda__trees_00__9}%
\includeinlinescaledsvg{.25}{.4}{lambda__trees_00__10}%
\includeinlinescaledsvg{.25}{.4}{lambda__trees_00__11}%
\end{center}\end{minipage}

We have ordered the trees and enumerated the lines to make the analysis clearer.

We need to generate all of these trees on the negative side of the equation. Let's take a look at the line (1). All of these trees share the same property, that the virtual leaf being pointed is child of \(l\)~node. Analogically at the line (2) pointed virtual leaves are children of \(s\) nodes. To generate such trees on the negative side we will use \(\tSection_{\{l\}}\) and \(\tSection_{\{s\}}\).

The transformation to resolve case of the line (1) is simple:

\begin{center}
    \includeinlinescaledsvg{.4}{.7}{lambda__transformations__002a}%
    \(\xRightarrow{-1}\)%
    \includeinlinescaledsvg{.4}{.7}{lambda__transformations__002b}%
\end{center}

\begin{lstlisting}
c.append_r_class_subtree_visitor(-1,                # negative node
    lambda kind, left, right:
        None if kind[0] is None else                # we ignore all virtual nodes
        ["l", kind] + left + right + [(None, "*")]  # for each actual node we add
                                                    # l node with right virtual
                                                    # node pointed as a parent
)
\end{lstlisting}

We just add an \(l\) node above each actual one and point its right child being virtual leaf.

It creates following trees:

\includeinlinesvg{.24}{lambda__trees_100__0_base}\(\xRightarrow{-1}\)\includeinlinesvg{.24}{lambda__trees_100__0}%
\includeinlinesvg{.24}{lambda__trees_100__1_base}\(\xRightarrow{-1}\)\includeinlinesvg{.24}{lambda__trees_100__1}%

\includeinlinesvg{.24}{lambda__trees_100__2_base}\(\xRightarrow{-1}\)\includeinlinesvg{.24}{lambda__trees_100__2}%
\includeinlinesvg{.24}{lambda__trees_100__3_base}\(\xRightarrow{-1}\)\includeinlinesvg{.24}{lambda__trees_100__3}%

It addresses all cases from the line (1), but also introduces an~incorrect lambda tree in which \(s\) node is the parent of \(l\) node, which cannot happen (i.e.~such tree does not correspond to valid lambda term). To mitigate this problem we will later use \(\tSection_{\{ls\}}\) and generate positive trees where \(s\) node is parent of \(l\) node.

To generate negative trees corresponding to the line (2), create following transformation for \(\tSection_{\{s\}}\).

\begin{center}
    \includeinlinescaledsvg{.4}{.7}{lambda__transformations__003a}%
    \(\xRightarrow{-1}\)%
    \includeinlinescaledsvg{.4}{.7}{lambda__transformations__003b}%
\end{center}

\begin{lstlisting}
c.append_r_class_right_parent_subtree_visitor(-1,
    lambda kind, parent_kind, left, right:
        None if kind[0] is not None else
        ["s", (None, "*"), parent_kind] + left + right
)
\end{lstlisting}

It generates exactly the trees we need:

\begin{center}
    \includeinlinescaledsvg{.24}{.4}{lambda__trees_110__0_base}%
    \(\xRightarrow{-1}\)%
    \includeinlinescaledsvg{.24}{.4}{lambda__trees_110__0}%

    \includeinlinescaledsvg{.24}{.4}{lambda__trees_110__1_base}%
    \(\xRightarrow{-1}\)%
    \includeinlinescaledsvg{.24}{.4}{lambda__trees_110__1}%
    \includeinlinescaledsvg{.24}{.4}{lambda__trees_110__2_base}%
    \(\xRightarrow{-1}\)%
    \includeinlinescaledsvg{.24}{.4}{lambda__trees_110__2}%
\end{center}

Next, we can take care of cases from line (3). To do so, we will use both \(\tSection_{\{l\}}\) and \(\tSection_{\{s\}}\).

\begin{center}
    \includeinlinescaledsvg{.4}{.7}{lambda__transformations__004a}%
    \(\xRightarrow{-1}\)%
    \includeinlinescaledsvg{.4}{.7}{lambda__transformations__004b}%
\end{center}

\begin{lstlisting}
c.append_r_class_right_parent_subtree_visitor(-1,
    lambda kind, parent_kind, left, right:
        None if kind[0] is not None else
        ["l", parent_kind] + [left[0] + ("*",)] + left[1:] + right + [None]
)
\end{lstlisting}

\begin{center}
    \includeinlinescaledsvg{.4}{.7}{lambda__transformations__005a}%
    \(\xRightarrow{-1}\)%
    \includeinlinescaledsvg{.4}{.7}{lambda__transformations__005b}%
\end{center}

\begin{lstlisting}
c.append_r_class_left_parent_subtree_visitor(-1,
    lambda kind, parent_kind, left, right:
        None if kind[0] is not None else
        ["l", parent_kind] + left + [right[0] + ("*",)] + right[1:] + [None]
)
\end{lstlisting}

\begin{center}
    \includeinlinescaledsvg{.4}{.7}{lambda__transformations__006a}%
    \(\xRightarrow{-1}\)%
    \includeinlinescaledsvg{.4}{.7}{lambda__transformations__006b}%
\end{center}

\begin{lstlisting}
c.append_r_class_left_parent_subtree_visitor(-1,
    lambda kind, parent_kind, left, right:
        None if kind[0] is not None else
        ["s", None, parent_kind] + left + [right[0] + ("*",)] + right[1:]
)
\end{lstlisting}

\begin{center}
    \includeinlinescaledsvg{.4}{.7}{lambda__transformations__007a}%
    \(\xRightarrow{-1}\)%
    \includeinlinescaledsvg{.4}{.7}{lambda__transformations__007b}%
\end{center}

\begin{lstlisting}
c.append_r_class_right_parent_subtree_visitor(-1,
    lambda kind, parent_kind, left, right:
        None if kind[0] is not None else
        ["s", None, parent_kind] + [left[0] + ("*",)] + left[1:] + right
)
\end{lstlisting}

Basically, for each virtual leaf not being root (i.e.~having left or right parent), we point it and add \(l\) or \(s\) respectively above the parent.

\includeinlinesvg{.32}{lambda__trees_111__1_base}%
\(\xRightarrow{-1}\)%
\includeinlinesvg{.32}{lambda__trees_101__0}%
\includeinlinesvg{.32}{lambda__trees_112__0}%

The last row can be generated using \(\tSection_{\{ao\}}\):

\begin{center}
    \includeinlinescaledsvg{.4}{.7}{lambda__transformations__008a}%
    \(\xRightarrow{-1}\)%
    \includeinlinescaledsvg{.4}{.7}{lambda__transformations__008b}%
\end{center}

\begin{lstlisting}
c.append_r_class_subtree_visitor(-1,
    lambda kind, left, right:
        None if kind[0] is None else
        ["a", "o", (None, "*"), None, kind] + left + right
)
\end{lstlisting}

\begin{center}
    \includeinlinescaledsvg{.4}{.7}{lambda__transformations__009a}%
    \(\xRightarrow{-1}\)%
    \includeinlinescaledsvg{.4}{.7}{lambda__transformations__009b}%
\end{center}

\begin{lstlisting}
c.append_r_class_subtree_visitor(-1,
    lambda kind, left, right:
        None if kind[0] is None else
        ["a", "o", None, (None, "*"), kind] + left + right
)
\end{lstlisting}

\begin{center}
    \includeinlinescaledsvg{.4}{.7}{lambda__transformations__010a}%
    \(\xRightarrow{-1}\)%
    \includeinlinescaledsvg{.4}{.7}{lambda__transformations__010b}%
\end{center}

\begin{lstlisting}
c.append_r_class_subtree_visitor(-1,
    lambda kind, left, right:
        None if kind[0] is None else
        ["a", kind] + left + right + ["o", (None, "*"), None]
)
\end{lstlisting}

\begin{center}
    \includeinlinescaledsvg{.4}{.7}{lambda__transformations__011a}%
    \(\xRightarrow{-1}\)%
    \includeinlinescaledsvg{.4}{.7}{lambda__transformations__011b}%
\end{center}

\begin{lstlisting}
c.append_r_class_subtree_visitor(-1,
    lambda kind, left, right:
        None if kind[0] is None else
        ["a", kind] + left + right + ["o", None, (None, "*")]
)
\end{lstlisting}

In other words, for each actual node, we add an \(a\) node as its left or right parent and add as a~second child \(o\) node as its sibling. We then point to the left or right virtual child of the newly created \(o\)~node:

\includeinlinesvg{.20}{lambda__trees_200__0_base}%
\(\xRightarrow{-1}\)%
\includeinlinesvg{.20}{lambda__trees_200__0}%
\includeinlinesvg{.20}{lambda__trees_201__0}%
\includeinlinesvg{.20}{lambda__trees_202__0}%
\includeinlinesvg{.20}{lambda__trees_203__0}%

After such transformations, we get following statistics:

\begin{lstlisting}
=== Classes stats ===
[x]  0,0:      16 of      16 | (k + 1) * Tk              []
[x]  1,0:     -10 of     -10 | - (2 * k + 1) * Tk        [l]
[ ]  1,1:      -9 of     -12 | - 2 * (k + 1) * Tk        [s]
[x]  2,0:      -4 of      -4 | - (4 * k) * Tk            [ao]
[ ]  2,1:       0 of       1 | (k) * Tk                  [ll]
[ ]  2,2:       0 of       6 | (4 * k + 2) * Tk          [ls]
[ ]  2,3:       0 of       2 | (k + 1) * Tk              [ss]
[ ]  3,0:       0 of       2 | 2 * (2 * k + 1) * Tk      [aos]
[x]  3,1:       0 of       0 | - (2 * k) * Tk            [lls]
[ ]  3,2:       0 of      -1 | - (2 * k + 1) * Tk        [lss]
[x]  4,0:       0 of       0 | k * Tk                    [llss]
=== Diff stats ===
1,0:     4     0
1,1:     3     0
\end{lstlisting}

Note that we have gotten rid of the \(\tSection_\emptyset\) in \verb|Diff stats|. It means that we have generated all negative counterpart of the trees generated by that.

If we increase \(n\) (i.e.~target size of the trees), this construction still works:

\begin{lstlisting}
=== Classes stats ===
[x]  0,0:   39050 of   39050 | (k + 1) * Tk              []
[x]  1,0:  -23237 of  -23237 | - (2 * k + 1) * Tk        [l]
[ ]  1,1:  -17799 of  -24460 | - 2 * (k + 1) * Tk        [s]
[x]  2,0:  -13728 of  -13728 | - (4 * k) * Tk            [ao]
[ ]  2,1:       0 of    3432 | (k) * Tk                  [ll]
[ ]  2,2:       0 of   14586 | (4 * k + 2) * Tk          [ls]
[ ]  2,3:       0 of    3861 | (k + 1) * Tk              [ss]
[ ]  3,0:       0 of    4620 | 2 * (2 * k + 1) * Tk      [aos]
[ ]  3,1:       0 of   -2156 | - (2 * k) * Tk            [lls]
[ ]  3,2:       0 of   -2310 | - (2 * k + 1) * Tk        [lss]
[ ]  4,0:       0 of     342 | k * Tk                    [llss]
=== Diff stats ===
1,0:  7602     0
1,1:  5832     0
2,0:  2280     0
\end{lstlisting}

The process can then be repeated for each section. We have chosen lexicographical order (i.e.~\(\tSection_\emptyset\); \(\tSection_{\{l\}}\); \(\tSection_{\{s\}}\); \(\tSection_{\{ao\}}\); \(\tSection_{\{ll\}}\); \(\ldots\)).

\subsection{Final result}%
\label{sub:final_result}

After multiple iterations, we managed to come up with following construction:

\begin{lstlisting}
c.define_class(0)

c.define_section(0)

c.append_r_class_subtree_visitor(1,
    lambda kind, left, right:
        None if kind[0] is not None else
        [kind + ("*", )] + left + right
)


c.define_class(1)

c.define_section(0)

c.append_r_class_subtree_visitor(-1,
    lambda kind, left, right:
        None if kind[0] is None else
        ["l", kind] + left + right + [(None, "*")]
)

c.append_r_class_right_parent_subtree_visitor(-1,
    lambda kind, parent_kind, left, right:
        None if kind[0] is not None else
        ["l", parent_kind] + [left[0] + ("*",)] + left[1:] + right + [None]
)

c.append_r_class_left_parent_subtree_visitor(-1,
    lambda kind, parent_kind, left, right:
        None if kind[0] is not None else
        ["l", parent_kind] + left + [right[0] + ("*",)] + right[1:] + [None]
)

c.define_section(1)

c.append_r_class_right_parent_subtree_visitor(-1,
    lambda kind, parent_kind, left, right:
        None if kind[0] is not None else
        ["s", (None, "*"), parent_kind] + left + right
)

c.append_r_class_left_parent_subtree_visitor(-1,
    lambda kind, parent_kind, left, right:
        None if kind[0] is not None else
        ["s", None, parent_kind] + left + [right[0] + ("*",)] + right[1:]
)

c.append_r_class_right_parent_subtree_visitor(-1,
    lambda kind, parent_kind, left, right:
        None if kind[0] is not None else
        ["s", None, parent_kind] + [left[0] + ("*",)] + left[1:] + right
)


c.define_class(2)

c.define_section(0)

c.append_r_class_subtree_visitor(-1,
    lambda kind, left, right:
        None if kind[0] is None else
        ["a", "o", (None, "*"), None, kind] + left + right
)

c.append_r_class_subtree_visitor(-1,
    lambda kind, left, right:
        None if kind[0] is None else
        ["a", "o", None, (None, "*"), kind] + left + right
)

c.append_r_class_subtree_visitor(-1,
    lambda kind, left, right:
        None if kind[0] is None else
        ["a", kind] + left + right + ["o", (None, "*"), None]
)

c.append_r_class_subtree_visitor(-1,
    lambda kind, left, right:
        None if kind[0] is None else
        ["a", kind] + left + right + ["o", None, (None, "*")]
)

c.define_section(1)

c.append_r_class_subtree_visitor(1,
    lambda kind, left, right:
        None if kind[0] is None else
        ["l", "l", kind] + left + right + [(None, "*"), None]
)

c.define_section(2)

c.append_r_class_right_parent_subtree_visitor(1,
    lambda kind, parent_kind, left, right:
        None if kind[0] is not None else
        ["l", "s", (None, "*"), parent_kind] + left + right + [None]
)

c.append_r_class_subtree_visitor(1,
    lambda kind, left, right:
        None if kind[0] is None else
        ["s", None, "l", kind] + left + right + [(None, "*")]
)

c.append_r_class_right_parent_subtree_visitor(1,
    lambda kind, parent_kind, left, right:
        None if kind[0] is not None else
        ["s", None, "l", parent_kind] + left + right + [(None, "*")]
)

c.append_r_class_right_parent_subtree_visitor(1,
    lambda kind, parent_kind, left, right:
        None if kind[0] is not None else
        ["s", None, "l", parent_kind] + [left[0] + ("*", )] + left[1:] + right + [None]
)

c.append_r_class_left_parent_subtree_visitor(1,
    lambda kind, parent_kind, left, right:
        None if kind[0] is not None else
        ["s", None, "l", parent_kind] + left + [right[0] + ("*", )] + right[1:] + [None]
)


c.define_section(3)

c.append_r_class_right_parent_subtree_visitor(1,
    lambda kind, parent_kind, left, right:
        None if kind[0] is not None else
        ["s", None, "s", (None, "*"), parent_kind] + left + right
)


c.define_class(3)

c.define_section(0)

c.append_r_class_right_parent_subtree_visitor(1,
    lambda kind, parent_kind, left, right:
        None if kind[0] is not None else
        ["s", None, "a", "o", (None, "*"), None, parent_kind] + left + right
)

c.append_r_class_right_parent_subtree_visitor(1,
    lambda kind, parent_kind, left, right:
        None if kind[0] is not None else
        ["s", None, "a", "o", None, (None, "*"), parent_kind] + left + right
)

c.append_r_class_right_parent_subtree_visitor(1,
    lambda kind, parent_kind, left, right:
        None if kind[0] is not None else
        ["s", None, "a", parent_kind] + left + right + ["o", (None, "*"), None]
)

c.append_r_class_right_parent_subtree_visitor(1,
    lambda kind, parent_kind, left, right:
        None if kind[0] is not None else
        ["s", None, "a", parent_kind] + left + right + ["o", None, (None, "*")]
)

c.define_section(1)

c.append_r_class_right_parent_subtree_visitor(-1,
    lambda kind, parent_kind, left, right:
        None if kind[0] is not None else
        ["s", None, "l", "l", parent_kind] + left + right + [(None, "*"), None]
)

c.append_r_class_right_parent_subtree_visitor(-1,
    lambda kind, parent_kind, left, right:
        ["s", None, "l", "l", parent_kind] + left + right + [(None, "*"), None]
)


c.define_section(2)

c.append_r_class_right_parent_subtree_visitor(-1,
    lambda kind, parent_kind, left, right:
        None if kind[0] is not None else
        ["s", None, "s", None, "l", parent_kind] + left + right + [(None, "*")]
)

c.append_r_class_right_parent_subtree_visitor(-1,
    lambda kind, parent_kind, left, right:
        None if kind[0] is not None else
        ["s", None, "l", "s", (None, "*"), parent_kind] + left + right + [None]
)


c.define_class(4)

c.define_section(0)

c.append_r_class_right_parent_subtree_visitor(1,
    lambda kind, parent_kind, left, right:
        None if kind[0] is not None else
        ["s", None, "s", None, "l", "l", parent_kind] + left + right + [(None, "*"), None]
)
\end{lstlisting}

It resulted in final statistics:

\begin{lstlisting}
=== Classes stats ===
[x]  0,0:   39050 of   39050 | (k + 1) * Tk              []
[x]  1,0:  -23237 of  -23237 | - (2 * k + 1) * Tk        [l]
[ ]  1,1:  -17799 of  -24460 | - 2 * (k + 1) * Tk        [s]
[x]  2,0:  -13728 of  -13728 | - (4 * k) * Tk            [ao]
[x]  2,1:    3432 of    3432 | (k) * Tk                  [ll]
[ ]  2,2:   10833 of   14586 | (4 * k + 2) * Tk          [ls]
[ ]  2,3:    1770 of    3861 | (k + 1) * Tk              [ss]
[ ]  3,0:    2280 of    4620 | 2 * (2 * k + 1) * Tk      [aos]
[ ]  3,1:   -1648 of   -2156 | - (2 * k) * Tk            [lls]
[ ]  3,2:   -1140 of   -2310 | - (2 * k + 1) * Tk        [lss]
[ ]  4,0:     187 of     342 | k * Tk                    [llss]
=== Diff stats ===
\end{lstlisting}

As you can see, \verb|Diff stats| are empty, so we managed to generate expected trees on both sides of the equation.

You may also note that we did not generate all trees suggested by the recurrence. This is due to one cheat we used.

\subsection{The cheat}%
\label{sub:the_cheat}

You may remember our assumptions in Section~\ref{sub:the_framework} about the discrimination cases. We allowed ourselves to apply visitors only to one of the left and right children, which is not handled by our holonomic recurrence as the number of such nodes does not depend only on \(n\).

To fix this, one can for example add construction for \(\tSection_{\{s\}}\) to generate all expected trees from it and then repeat the iteration in order to generate matching trees on the other side of the equation. We can see that we used \(2\) constructions visiting nodes with right parent and only one visiting node having left parent, so our missing construction should also visit nodes with left parent.

We already have defined following transformations:

\begin{center}
    \includeinlinescaledsvg{.4}{.7}{lambda__transformations__003a}%
    \(\xRightarrow{-1}\)%
    \includeinlinescaledsvg{.4}{.7}{lambda__transformations__003b}%
\end{center}

\begin{lstlisting}
c.append_r_class_right_parent_subtree_visitor(-1,
    lambda kind, parent_kind, left, right:
        None if kind[0] is not None else
        ["s", (None, "*"), parent_kind] + left + right
)
\end{lstlisting}

\begin{center}
    \includeinlinescaledsvg{.4}{.7}{lambda__transformations__012a}%
    \(\xRightarrow{-1}\)%
    \includeinlinescaledsvg{.4}{.7}{lambda__transformations__012b}%
\end{center}

\begin{lstlisting}
c.append_r_class_left_parent_subtree_visitor(-1,
    lambda kind, parent_kind, left, right:
        None if kind[0] is not None else
        ["s", None, parent_kind] + left + [right[0] + ("*",)] + right[1:]
)
\end{lstlisting}

\begin{center}
    \includeinlinescaledsvg{.4}{.7}{lambda__transformations__013a}%
    \(\xRightarrow{-1}\)%
    \includeinlinescaledsvg{.4}{.7}{lambda__transformations__013b}%
\end{center}

\begin{lstlisting}
c.append_r_class_right_parent_subtree_visitor(-1,
    lambda kind, parent_kind, left, right:
        None if kind[0] is not None else
        ["s", None, parent_kind] + [left[0] + ("*",)] + left[1:] + right
)
\end{lstlisting}

An example of transformation we would like to add can be:

\begin{center}
    \includeinlinescaledsvg{.4}{.7}{lambda__transformations__014a}%
    \(\xRightarrow{-1}\)%
    \includeinlinescaledsvg{.4}{.7}{lambda__transformations__014b}%
\end{center}

\begin{lstlisting}
c.append_r_class_left_parent_subtree_visitor(-1,
    lambda kind, parent_kind, left, right:
        None if kind[0] is not None else
        ["s", (None, "*"), parent_kind] + left + right
)
\end{lstlisting}

Please note that it is only an~example of missing construction. It is just the simplest one we could think about, and in that case, a~simple solution is most often the correct one.

\section{Further work}%
\label{sec:further_work}

First thing to be done, that comes to mind immediately, is to finish the construction eliminating the cheat mentioned in Section~\ref{sub:the_cheat}.

You may also note that we are operating on specific \(N\). The probability that we miss some corner case is low, but even if we check large \(N\) values, it is always present. The formal algebraic proof would make us sure that the construction works. We would prioritize finishing the construction first.

The framework also is not perfect. It highly depends on the operator's intuition. We believe that the solving process can be automated. The simplest method that comes to mind is simply creating heuristic rules that then will be used by the framework to guess at least partial solutions.

With the proper interpretation, it may be possible to create a~sampling algorithm similar to Rémy's one for binary trees~\cite{remy,note}. It would have to take under consideration that some trees are being removed by other classes, so the probabilities would have to be set carefully.

We believe that one day all of that will be done or proven to be impossible.

\clearpage

\printbibliography%

\end{document}
